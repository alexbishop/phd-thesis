\chapter{Polyhedral Sets and Polyhedrally Constrained Languages}\label{chp:polyhedral}

In this chapter we introduce the family of \emph{polyhedrally constrained languages} which we use in the proof of \cref{thm:geodesic-growth}.
This family of languages is a generalisation of linearly constrained languages, as in \cref{defn:linearly-constrained-language}.
It is a result of \textcite{massazza1993} that the generating function of linearly constrained languages is holonomic (see~\cref{prop:lcl-is-holonomic}).
In \cref{prop:polyhedrally-constrained-is-holonomic}, we show that the family of polyhedrally constrained languages also have holonomic (multivariate) generating functions.

\Textcite{benson1983} introduced the concept of \emph{polyhedral sets} to compute the volume growth of virtually abelian groups, in particular, for each virtually abelian group \citeauthor{benson1983} constructed a polyhedral set whose (volume) generating function is the volume growth series of the group.
In \cref{sec:virtually-abelian-groups} we apply a similar argument to construct a \emph{polyhedrally constrained language} whose generating function is the geodesic growth series of a virtually abelian group.
We define and study the class of polyhedral sets in \cref{sec:polyhedral-sets}, and the family of polyhedrally constrained languages in \cref{sec:polyhedrally-constrained-languages}.

\section{Polyhedral Sets}\label{sec:polyhedral-sets}

A \emph{polyhedral set}, as we see in \cref{defn:polyhedral-set}, is a subset of $\mathbb{Z}^n$ that encodes the integer solutions to finitely many systems of linear equations, inequalities and congruences.
The class of polyhedral sets is closed under Boolean expressions, Cartesian products, and (inverse) mapping by integer affine transformation (see \cref{prop:affine-transforms-of-polyhedral-sets,prop:closure-properties-of-polyhedral-sets}).
The class of polyhedral sets and their closure properties are essential to our study of the language of geodesics for virtually abelian groups in \cref{sec:virtually-abelian-groups}.

\begin{definition}\label{defn:polyhedral-set}
A subset $\mathcal{E} \subseteq \mathbb{Z}^m$ is called an \emph{elementary region} if it can be expressed as
\[
	\left\{
		z \in \mathbb{Z}^m
	\, \middle\vert \,
		a\cdot z = b
	\right\},
	%
	\ 
	%
	\left\{
		z \in \mathbb{Z}^m
	\, \middle\vert \,
		a\cdot z > b
	\right\}
	%
	\text{ or } 
	%
	\left\{
		z \in \mathbb{Z}^m
	\, \middle\vert \,
		a\cdot z \equiv b\ (\bmod\ c)
	\right\}
\]
for some $a \in \mathbb{Z}^m$ and $b,c\in \mathbb{Z}$ with $c > 0$.
A \emph{basic polyhedral set} is a finite intersection of elementary regions;
and a \emph{polyhedral set} is a finite disjoint union of basic polyhedral sets.
\end{definition}

From this definition we see that $\emptyset$ and $\mathbb{Z}^m$ are elementary regions, and that $\mathbb{N}^m$ is a basic polyhedral set.
In \cref{prop:closure-properties-of-polyhedral-sets} we see that the class of polyhedral sets is closed under Boolean expressions and Cartesian products.

\begin{proposition}[Proposition~13.1~and~Remark~13.2~in~\cite{benson1983}]\label{prop:closure-properties-of-polyhedral-sets}
	The class of polyhedral subsets of $\mathbb{Z}^m$ is{\tiny } closed under finite union, finite intersection and set difference.
	Moreover, the class of polyhedral sets is closed under Cartesian product.
\end{proposition}

A map $E \colon \mathbb{Z}^m \to \mathbb{Z}^n$ is an \emph{integer affine transform} if it can be written as $E(v) = vA + b$ where $A \in \mathbb{Z}^{m \times n}$ is a matrix and $b \in \mathbb{Z}^n$ is a vector.
In \cref{prop:affine-transforms-of-polyhedral-sets} we see that the class of polyhedral sets is closed under (inverse) mapping by integer affine transformations.

\begin{proposition}[Propositions~13.7~and~13.8~in~\cite{benson1983}]\label{prop:affine-transforms-of-polyhedral-sets}
	Suppose that $\mathcal{P} \subseteq \mathbb{Z}^m$ and $\mathcal{Q} \subseteq \mathbb{Z}^n$ are polyhedral sets, and $E\colon\mathbb{Z}^m\to\mathbb{Z}^n$ is an integer affine transform.
	Then, $E(\mathcal{P})$ and $E^{-1}(\mathcal{Q})$ are both polyhedral sets.
\end{proposition}

Notice that our definition of $n$-atoms and $n$-constraints in \cref{defn:n-constraints} is similar to that of elementary regions and polyhedral sets, respectively, without modular arithmetic.
From the closure properties in \cref{prop:closure-properties-of-polyhedral-sets} we see that $n$-constraints form a subclass of the polyhedral subsets of $\mathbb{Z}^n$.
It can be verified by the reader that, for each $n \geqslant 1$,
\[
	\{
		(x,0,0,\ldots,0) \in \mathbb{Z}^n
	\mid
		x \equiv 0\ (\bmod\ 2)
	\}
\]
is a polyhedral set but not an $n$-constraint, and thus the class of $n$-constraints form a proper subclass of the polyhedral subsets of $\mathbb{Z}^n$.
In the following section, we generalise the family of linearly constrained languages, defined in \cref{defn:linearly-constrained-language}, to the family of \emph{polyhedral constrained languages} which we use in the proof of \cref{thm:geodesic-growth}.

\section{Polyhedrally Constrained Languages}\label{sec:polyhedrally-constrained-languages}

In \cref{sec:constrained-languages}, we saw that a constrained language, $L(U,\mathcal{C})$, is the intersection of an unambiguous context-free language $U \subseteq \Sigma$ and the set of words whose Parikh images belong to a set $\mathcal{C} \subseteq \mathbb{Z}^{|\Sigma|}$.
Moreover, we defined the family of linearly constrained languages in \cref{defn:linearly-constrained-language} as the constrained languages where $\mathcal{C}$ is a $|\Sigma|$-constraint.
In this section we generalise this definition to the family of \emph{polyhedrally constrained languages} as follows.

\begin{definition}\label{defn:polyhedrally-constrained-language}
	A language is \emph{polyhedraly constrained} if it can be written as
	\[
		L(U,\mathcal{P})
		=
		\left\{
			w \in U \subseteq \Sigma^*
		\mid
			\Parikh_\Sigma(w) \in \mathcal{P}
		\right\}
	\]
	where $U$ is an unambiguous context-free language, and $\mathcal{P}$ is a polyhedral set.
\end{definition}

In \cref{sec:virtually-abelian-groups} we will not require the full power of polyhedrally constrained languages, in particular, we only require polyhedrally constrained languages $L(U,\mathcal{P})$ where $U$ is a regular language.
It can then be shown that such languages form a subfamily of the \emph{RCM languages} introduced by \textcite{castiglione2017}, moreover, they showed that the single-variable generating functions of these languages are holonomic.
In \cref{prop:polyhedrally-constrained-is-holonomic}, we show that the multivariate generating function of each polyhedrally constrained language is holonomic.
We make use of this characterisation in the proof of \cref{thm:geodesic-growth}.

\begin{proposition}\label{prop:polyhedrally-constrained-is-holonomic}
	The multivariate generating function of a polyhedrally constrained language is holonomic.
\end{proposition}

\begin{proof}	
	Let $L(U,\mathcal{P}) \in \Sigma^*$ be a polyhedrally constrained language.
	From the definition of polyhedral sets, we may decompose $\mathcal{P} \subseteq \mathbb{Z}^{|\Sigma|}$ into a union of finitely many disjoint basic polyhedral sets
	$\mathcal{P} = \bigcup_{i=1}^L \mathcal{B}_i$.
	Moreover, each such basic polyhedral set $\mathcal{B}_i \subseteq \mathbb{Z}^{|\Sigma|}$ can be written as a finite intersection of elementary regions
	\begin{multline*}
		\mathcal{B}_i
		=
		\bigcap_{j = 1}^{K_{i,1}}
		\{
			v \in \mathbb{Z}^{|\Sigma|}
		\mid
			\alpha_{i,j} \cdot v = \beta_{i,j}
		\}
		\cap
		\bigcap_{j = 1}^{K_{i,2}}
		\{
			v \in \mathbb{Z}^{|\Sigma|}
		\mid
			\xi_{i,j} \cdot v > \lambda_{i,j}
		\}
		\\\cap
		\bigcap_{j = 1}^{K_{i,3}}
		\{
			v \in \mathbb{Z}^{|\Sigma|}
		\mid
			\zeta_{i,j} \cdot v \equiv \eta_{i,j}\ (\bmod\ \theta_{i,j})
		\}
	\end{multline*}
	where each $\alpha_{i,j},\xi_{i,j},\zeta_{i,j} \in \mathbb{Z}^{|\Sigma|}$, each $\beta_{i,j},\lambda_{i,j}, \eta_{i,j} \in \mathbb{Z}$, and $\theta_{i,j} \in \mathbb{N}_+$.
	
	From the definition of constrained language we see that $L(U,\mathcal{P})$ is the union of disjoint polyhedrally constrained languages $L(U,\mathcal{B}_i)$.
	We see that if each $L(U,\mathcal{B}_i)$ has a multivariate generating function of $f_i(x_1,x_2,\ldots,x_{|\Sigma|})$, then the multivariate generating function for $L(U,\mathcal{P})$ is given by
	\[
		f(x_1,x_2,\ldots,x_{|\Sigma|})
		=
		\sum_{i=1}^L f_i(x_1,x_2,\ldots,x_{|\Sigma|}).
	\]
	
	For each basic polyhedral set $\mathcal{B}_i$, we introduce a $|\Sigma|$-constraint
	\[
		\mathcal{C}_i
		=
		\bigcap_{j = 1}^{K_{i,1}}
		\{
			v \in \mathbb{Z}^{|\Sigma|}
		\mid
			\alpha_{i,j} \cdot v = \beta_{i,j}
		\}
		\cap
		\bigcap_{j = 1}^{K_{i,2}}
		\{
			v \in \mathbb{Z}^{|\Sigma|}
		\mid
			\xi_{i,j} \cdot v > \lambda_{i,j}
		\},
	\]
	and a monoid homomorphism $\varphi_i \colon \Sigma^* \to \prod_{j=1}^{K_{i,3}} (\mathbb{Z}/\theta_{i,j}\mathbb{Z})$ such that
	\[
		\varphi_i(w)
		=
		(
			\zeta_{i,1} \cdot \Parikh_\Sigma(w),\,
			\zeta_{i,2} \cdot \Parikh_\Sigma(w),\,
			\ldots,\,
			\zeta_{i,K_{i,3}} \cdot \Parikh_\Sigma(w)
		);
	\]
	moreover, we write $R_i \in \Sigma^*$ for the inverse image
	\[
		R_i =
		\varphi_i^{-1}(\{ (\eta_{i,1},\eta_{i,2},\ldots,\eta_{i,K_{i,3}}) \}).
	\]
	
	Each language $R_i \in \Sigma^*$ is expressed as the inverse image of a subset of a finite monoid.
	From \cite[Theorem~1]{rabin1959} we see that each $R_i$ is a regular language, in particular, for each $R_i$ we may construct a finite-state automaton with states given by the set $\prod_{j=1}^{K_{i,3}} (\mathbb{Z}/\theta_{i,j}\mathbb{Z})$, initial state given by $(0,\ldots,0)$, an accepting state of $(\eta_{i,1},\eta_{i,2},\ldots,\eta_{i,K_{i,3}})$, and a transition $v \to^\sigma v'$ for each state $v$ and letter $\sigma \in \Sigma$ where $v' = v + \varphi_i(\sigma)$.
	Moreover, since the class of unambiguous context-free grammar is closed under intersection with regular language (see Theorem~6.4.1 on p.~197 of \cite{harrison1978}), we see that each $L(U \cap R_i,\mathcal{C}_i) = L(U,\mathcal{B}_i)$ is linearly constrained as in \cref{defn:linearly-constrained-language}.
	From \cref{prop:lcl-is-holonomic}, we see that each $f_i(x_1,x_2,\ldots,x_{|\Sigma|})$ is holonomic.
	
	From \cref{lemma:holonomic-closure-properties}, holonomic functions are closed under addition, and thus the multivariate generating function of $L(U, \mathcal{P})$ is holonomic.
\end{proof}
