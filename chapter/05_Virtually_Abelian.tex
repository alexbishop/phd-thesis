\chapter{Virtually Abelian Groups}\label{chapter:polynomial-geodesic-growth}%
\label{sec:virtually-abelian-groups}

It is a well-known result of \textcite{gromov1981} that a group has polynomial volume growth if and only if it is virtually nilpotent.
Moreover, from the work of \textcite{bass1972} we know that virtually nilpotent groups have polynomial volume of integer degrees.
\Textcite{bridson2012} asked if there is an analogous classification for groups with polynomial geodesic growth and if there exists a group with intermediate geodesic growth.
Towards these questions they provided a sufficient condition, given in \cref{lem:bbes-main-theorem}, for a virtually abelian group to have polynomial geodesic growth, and furnished an example of a virtually $\mathbb{Z}^2$ group, given in \cref{eq:virtually-z2-example}, with polynomial geodesic growth.
Before this virtually $\mathbb{Z}^2$ example, the only groups known to have polynomial geodesic growth were virtually cyclic.
In this chapter, we take the next step towards a classification of polynomial geodesic growth by characterising the geodesic growth series for all virtually abelian groups with respect to any finite weighted monoid generating sets.
%We then see in \cref{thm:geodesic-growth} that this implies that the geodesic growth is either exponential, or polynomial with a rational geodesic growth series.

\begin{lemma}[Theorem~1 in \cite{bridson2012}]\label{lem:bbes-main-theorem}
	Let $G$ be a finitely-generated group.
	If there is an element $g \in G$ whose normal closure is a finite-index abelian subgroup of $G$, then $G$ has polynomial geodesic growth with respect to some generating set.
\end{lemma}

In \cite{bridson2012}, it was shown that the virtually $\mathbb{Z}^2$ group
\begin{equation}\label{eq:virtually-z2-example}
	\left\langle
		a,b,t
	\mid
		[a,b] = t^2 = 1,\,
		a^t = b
	\right\rangle
\end{equation}
has polynomial geodesic growth with respect to the generating set $\{a,a^{-1},t\}$.
This group was introduced by Cannon~\cite[Example~4.4.1~on~p.~97]{epstein1992} as an example of a group that is \textit{short-lex automatic} with respect to one, but not all, generating sets.
Moreover, it was shown in Example~4.4.2~on~page~98 of~\cite{epstein1992} that this group has a generating set for which the geodesics do not form a regular language.
In \cref{thm:bounded automata is ET0L}, we show that the language of geodesics for each virtually abelian group is blind multicounter for every generating set.

We may generalise the construction given in \cref{eq:virtually-z2-example} to show that for any finitely-generated abelian group, $A$, there is a virtually-$A$ group with polynomial geodesic growth.
Let $A$ be a finitely-generated abelian group, then from the classification of finitely-generated abelian groups we see that $A = F \times \mathbb{Z}^n$ for some $n$ and finite group $F$.
Let $x_1,x_2,\ldots,x_n$ be the standard basis for the $\mathbb{Z}^n$ subgroup of $A$.
Let
\begin{equation}\label{eq:virt-abel-polynomial}
	B
	=
	\left\langle
		A, t
	\ \middle\vert\ 
		[F,t]=1,\ 
		x_i^t = x_{i+1}\text{ for each }i < n,\ 
		t^n = 1
	\right\rangle.
\end{equation}
We see that $B$ contains $A$ as a subgroup of index $n$; and thus $B$ contains $\mathbb{Z}^n$ as a subgroup of index $n |F|$.
Moreover, the normal closure of $x_1$ in $B$ is the finite-index free-abelian subgroup $\mathbb{Z}^n$.
From \cref{lem:bbes-main-theorem}, we see that $B$ has polynomial geodesic growth with respect to some generating set.

Benson~\cites{benson1983} showed that the volume growth series for virtually abelian groups is rational with respect to any finite (weighted monoid) generating set.
This result was generalised by \textcite{evetts2019} who showed that the coset, subgroup, and conjugacy growth series of a virtually abelian group is rational with respect to any finite (weighted monoid) generating set.
In \crefrange{sec:patterned-words}{sec:patterned-words/properties} %
we modify the methods of Benson, and provide a characterisation of the geodesic growth series in \cref{sec:geodesic-growth} by combining this with our result on polyhedrally constrained language given in \cref{prop:polyhedrally-constrained-is-holonomic}.

\section{Patterned Words}\label{sec:patterned-words}

Let $G$ be a virtually abelian group that is generated as a monoid by some finite weighted generating set $S$.
It is known that $G$ contains a finite-index normal subgroup that is isomorphic to $\mathbb{Z}^n$ for some $n$.
This follows as $G$ must contain an abelian subgroup $A$ of finite index, then from the classification of finitely generated abelian groups we see that $G$ must contain a group $H$ that is isomorphic to $\mathbb{Z}^n$.
We may then obtain a normal subgroup from the core of $H$ as $\bigcap_{g\in G} H^g$.
This subgroup will be finite index in $G$, and is free abelian as it is a subgroup of a free abelian group $H$~\cite[100-1]{robinson1996}.
Without loss of generality, we assume that $\mathbb{Z}^n \triangleleft G$ with $d = [G:\mathbb{Z}^n]$.
We fix a set of coset representatives $T = \{t_1=1,t_2,\ldots,t_d\}$ for $\mathbb{Z}^n$ in $G$.
We then write elements of $G$ in the normal form $g = z \cdot t$ where $z \in \mathbb{Z}^n$ and $t \in T$.

\begin{definition}\label{defn:phi-rho}
	Let $\psi\colon G \to \mathbb{Z}^n$ and $\rho\colon G \to T$ be the maps defined such that the normal form for $g \in G$ is given by $\psi(g) \cdot \rho(g)$.
\end{definition}

Benson~\cite{benson1983} showed that virtually abelian groups have rational volume growth series by demonstrating that each group element has at least one geodesic representative that can be expressed as a \emph{patterned word}, where the set of such patterned words is then studied using the theory of polyhedral sets.
In this section we modify these arguments to study the set of all geodesic words in $S^*$, in particular, we describe \cref{algo:word-shuffling} which converts words in $S^*$ to \emph{patterned words} which represent the same group element with the same weight.
In \cref{sec:patterned-words/properties} we compute the weight and group element of patterned words, and describe the patterned words which correspond to geodesics.

We begin by defining two finite sets of words $Y,P \subseteq S^*$ as follows.

\begin{definition}\label{defn:sets-Y-P}
	From the generating set $S$ and the normal subgroup $\mathbb{Z}^n \triangleleft G$ with finite index $d = [G : \mathbb{Z}^n]$, we define the sets
	\begin{align*}
		Y
		&=
			\{
				\sigma \in S^*
			\mid
				1 \leqslant |\sigma|_S \leqslant d
				\ \,\mathrm{and}\,\ 
				\overline{\sigma} \in \mathbb{Z}^n
			\}
		\ \ \text{and}
		\\
		P
		&=
		\{
				\sigma \in S^*
			\mid
				1 \leqslant |\sigma|_S \leqslant d - 1
				\ \,\mathrm{and}\,\ 
				\overline{\sigma} \notin \mathbb{Z}^n
			\},
	\end{align*}
	and we fix a labelling $\{y_1,y_2,\ldots,y_m\} = Y$ where $m = |Y|$.
\end{definition}

We define the sets $Y$ and $P$ as above so that we have the technical property given in \cref{lemma:factoring-words}.
We will find this property useful in the proof of \cref{lemma:map-delta} which is then used to construct \cref{algo:word-shuffling}.

\begin{lemma}\label{lemma:factoring-words}
	Suppose that $w \in S^*$ with $1 \leqslant |w|_S \leqslant d$ and $w \notin P$.
	Then, there is a factoring $w = \alpha \beta \delta$ with $\alpha \in P \cup \{ \varepsilon \}$, $\beta \in Y$ and $\delta \in S^*$.
	In particular, there is a unique choice of such a factoring for which $(|\alpha|_S, |\beta|_S) \in \mathbb{N}^2$ is minimal with respect to the lexicographic ordering on $\mathbb{N}^2$.
\end{lemma}

\begin{proof}
Let $w = w_1 w_2 \cdots w_k$ with $1 \leqslant k \leqslant d$ and $w \notin P$.

Notice that if we have at least one such factorisation, then there is a unique choice of such a factoring where $(|\alpha|_S, |\beta|_S) \in \mathbb{N}^2$ is minimal with respect to the lexicographic ordering on $\mathbb{N}^2$.
Thus, all that remains to be shown is that at least one such factoring $w = \alpha\beta\delta$ exists.

If $|w|_S < d$, then we have such a factorisation given by $\beta = w$, and $\alpha = \delta = \varepsilon$.
Thus, in the remainder of this proof we consider the case where $|w|_S = d$.

If $|w|_S = d$, then from the pigeonhole principle on the $d$ cosets, we see that there must be a nontrivial factor $b = w_i w_{i+1} \cdots w_j$ for which $\overline{b} \in \mathbb{Z}^n$.
Let $I \geqslant 1$ be the smallest value for which there is a $J \geqslant I$ with $\overline{w_{I} w_{I+1} \cdots w_J} \in \mathbb{Z}^n$, then let $\alpha = w_1 w_2 \cdots w_{I-1}$ and $\beta = w_{I} w_{I+1} \cdots w_J$.
From our choice of indices $I$ and $J$, we see that $\beta \in Y$, and either $\alpha = \varepsilon$ or $\overline{\alpha} \notin \mathbb{Z}^n$.
Moreover, we see that $|\alpha|_S = I-1 \leqslant d - 1$ and thus $\alpha \in P\cup \{\varepsilon\}$.
\end{proof}

Notice that $S \subseteq Y \cup P$, and thus $Y \cup P$ generates the group $G$.
We will see that for each word $\sigma \in S^*$, there is a word $w \in Y^*(PY^*)^k$, with $0 \leqslant k \leqslant d$, such that $w$ represents the same group element as $\sigma$ with the same weight.
We formalise this by defining \emph{patterns} and \emph{patterned words} as follows.

\begin{definition}[Patterned words]\label{defn:patterned-words}
	Let $\pi = \pi_1 \pi_2 \cdots \pi_k \in P^*$ be a word in the letters of $P$ with length $k = |\pi|_P \leqslant d$ for which each proper prefix belongs to a distinct coset, that is,
	\begin{equation}\label{eq:pattern-cosets}
		1 = \rho(\overline{\varepsilon}),\ 
		\rho(\overline{\pi_1}),\ 
		\rho(\overline{\pi_1\pi_2}),\ 
		\ldots,\ 
		\rho(\overline{\pi_1\pi_2\cdots\pi_{k-1}})
	\end{equation}
	are pairwise distinct; and let $v \in \mathbb{N}^{(k+1)m}$ be a vector where $m = |Y|$.
	Then we say that $\pi$ is a \emph{pattern} and that $(v,\pi)$ is a \emph{patterned word}.
	We then write
	\begin{equation*}
		v^\pi
		=
		\Big(
			y_1^{v_1}
			y_2^{v_2}
			\cdots
			y_m^{v_m}
		\Big)
		\pi_1
		\Big(
			y_1^{v_{m+1}}
			y_2^{v_{m+2}}
			\cdots
			y_m^{v_{2m}}
		\Big)
		\pi_2
		\cdots
		\pi_k
		\Big(
			y_1^{v_{k \cdot m+1}}
			y_2^{v_{k \cdot m+2}}
			\cdots
			y_m^{v_{(k+1)\cdot m}}
		\Big).
	\end{equation*}
	Notice that $\rho(\overline{\pi})$ is not included in (\ref{eq:pattern-cosets}).
	If $\rho(\overline{\pi})$ is also distinct from each coset representative in (\ref{eq:pattern-cosets}), then we say that $\pi$ is a \emph{strong pattern} and that $(v,\pi)$ is a \emph{strongly patterned word}.
\end{definition}

To simplify notation in later sections, we introduce the following sets.

\begin{definition}\label{defn:pattern-set}
	We write $\textsc{Patt} \subseteq P^*$ for the set of all patterns, and we write $\textsc{StrPatt} \subseteq \textsc{Patt}$ for the set of all strong patterns.
	Notice that $\textsc{Patt}$ and $\textsc{StrPatt}$ are finite, in particular, $|\textsc{Patt}| \leqslant |P|^{d+1}$.
\end{definition}

To simplify notation in \cref{sec:patterned-words/word-shuffling-algorithm}, we extend this as follows.

\begin{definition}[Extended Patterned Words]\label{defn:extended-special-form}
	If $(v,\pi)$ is a (strongly) patterned word, and $\sigma \in S^*$, then $((v,\pi),\sigma)$ is an \emph{extended (strongly) patterned word}.
\end{definition}

In \cref{algo:word-shuffling}, for each word $\sigma \in S^*$, we construct a finite sequence of extended patterned words that begins with $((\mathbf{0},\varepsilon),\sigma)$ and ends with an extended patterned word of the form $((v,\pi),\varepsilon)$.
Moreover, this sequence has the property that $v^\pi$ and $\sigma$ represent the same group element with the same weight.
To simplify notation, we define the following equivalence relation.

\begin{definition}\label{defn:equiv-relation}
	We define the equivalence relation $\simeq$ on $S^*$ such that, for each $w,\sigma \in S^*$, we have $w \simeq \sigma$ if and only if both $\overline{w} = \overline{\sigma}$ and $\omega(w) = \omega(\sigma)$.
\end{definition}

Notice that if we have a patterned word $(v,\pi)$ with $v^\pi \simeq \sigma$, then $\sigma$ is a geodesic if and only if the word $v^\pi$ is a geodesic.

\subsection{Word Shuffling}\label{sec:patterned-words/word-shuffling-algorithm}

In this section we construct \cref{algo:word-shuffling} which `shuffles' words of the form $\sigma \in S^*$ into patterned words $(v,\pi)$ which represent the same group element with the same weighted length.
In particular, for each word $\sigma$, we compute a finite sequence of extended patterned words
\begin{multline}\label{eq:patterned-sequence}
	((\mathbf{0},\varepsilon),\sigma)
	=
	((u^{(1)},\tau^{(1)}),\sigma^{(1)}),
	((u^{(2)},\tau^{(2)}),\sigma^{(2)}),\\
	\ldots,
	((u^{(q)},\tau^{(q)}),\sigma^{(q)})
	=
	((v,\pi),\varepsilon),
\end{multline}
such that
\[
	(u^{(i)})^{\tau^{(i)}} \sigma^{(i)}
	\simeq
	(u^{(i+1)})^{\tau^{(i+1)}}\sigma^{(i+1)}
	\quad
	\text{and}
	\quad
	|\sigma^{(i)}|_S > |\sigma^{(i+1)}|_S
\]
for each $i$.
Notice that $v^\pi \simeq \sigma$, and $q \leqslant |\sigma|_S+1$ where $q$ is the length of the sequence in (\ref{eq:patterned-sequence}).
From (\ref{eq:patterned-sequence}), we define $\mathrm{Shuffle}(\sigma) = (v,\pi)$ where the patterned word $(v,\pi)$ has the property that $v^\pi \simeq \sigma$.

The idea of \cref{algo:word-shuffling} is to compute each $((u^{(i+1)},\tau^{(i+1)}),\sigma^{(i+1)})$ from its previous extended patterned word $((u^{(i)},\tau^{(i)}),\sigma^{(i)})$ by replacing a bounded-length prefix of $\sigma^{(i)}$ with a strictly shorter word, adding at most a unit vector to $u^{(i)}$, and adding at most one letter to $\tau^{(i)}$.
In order to describe our algorithm, we introduce the following additional notation.

Recall that $d = [G : \mathbb{Z}^n]$ is the index of the $\mathbb{Z}^n$ normal subgroup of $G$.
For each word $\sigma \in S^*$, we fix a bounded-length prefix as follows.

\begin{definition}[Prefixes]\label{rmk:short-prefix}
	We write $\mathrm{Prefix} \colon S^* \to S^*$ for the function which computes the prefix of a word of length at most $d$, that is, $\mathrm{Prefix}(\sigma) = \sigma_1 \sigma_2 \cdots \sigma_q$ where $q = \min(d,|\sigma|_S)$.
	Notice that if $w = \mathrm{Prefix}(\sigma)$ with $|w|_S < d$, then $\sigma = w$.
\end{definition}

In sequence (\ref{eq:patterned-sequence}), each word $\sigma^{(i+1)}$ is obtained from $\sigma^{(i)}$ by replacing the prefix $w^{(i)} = \mathrm{Prefix}(\sigma^{(i)})$ with a strictly shorter word $w^{(i)\prime}$.
We write these prefix replacements using the following notation.

\begin{definition}[Prefix Replacements]
Let $\sigma \in S^*$ be a word which factors as $\sigma = w\zeta$ where $w,\zeta \in S^*$, then for each word $w' \in S^*$ we write $(w \mapsto w') \cdot \sigma = w'\zeta$ which we call a \emph{prefix replacement}.
We write a sequence of replacements as
\[
	(w_n \mapsto w'_n)
	\cdots
	(w_2 \mapsto w'_2)
	(w_1 \mapsto w'_1)
	\cdot \sigma
\]
where replacements are composed right-to-left.
Notice that if $\sigma' = (w \mapsto w')\cdot \sigma$, then $\omega(\sigma') = \omega(\sigma) - \omega(w) + \omega(w')$ where $\omega \colon S^* \to \mathbb{N}$ is the weight function.
\end{definition}

To understand how prefix replacements are composed, consider the following.

\begin{example}\label{ex:prefix-replacement-sequence}
We have the sequence of replacements
\begin{equation}\label{eq:prefix-replacement-example}
	(c \mapsto dc)
	(ba \mapsto cb)
	(\varepsilon \mapsto b)
	\cdot
	az
	=
	dcbz.
\end{equation}
Notice that if $(w \mapsto w')\cdot \sigma$ is defined, then we have $\sigma = (w' \mapsto w)(w \mapsto w')\cdot \sigma$, that is, each prefix replacement has an inverse.
For example, from the sequence of prefix replacements given in (\ref{eq:prefix-replacement-example}), we see that
\[
	az =
	(b \mapsto \varepsilon)
	(cb \mapsto ba)
	(dc \mapsto c)
	\cdot
	dcbz.
\]
Thus, we may compute the inverse of a sequence of prefix replacements.
\end{example}

For each pattern $\pi$, we write $\mathcal{N}_\pi$ for the set of all vectors $v$ for which $(v,\pi)$ is a patterned word, as defined in \cref{defn:patterned-words}.
We introduce the following notation to simplify the description of our algorithm.

\begin{definition}\label{defn:standard-basis-elements}
For each pattern $\pi = \pi_1 \pi_2 \cdots \pi_k \in P^*$, we write $\mathcal{Z}_\pi$ and $\mathcal{N}_\pi$ for the sets $\mathbb{Z}^{(k+1)m}$ and $\mathbb{N}^{(k+1)m}$, respectively, where $m = |Y|$.
Moreover, for each $i \in \{ 1,2,\ldots,\dim(\mathcal{Z}_\pi) \}$ we write $e_{\pi,i}$ for the $i$-th standard basis element of $\mathcal{Z}_\pi$ and $e_{\pi,\varnothing} = \mathbf{0} \in \mathcal{Z}_\pi$ for the zero vector of $\mathcal{Z}_\pi$.
\end{definition}

When computing (\ref{eq:patterned-sequence}), it may be the case that $|\tau^{(i)}|_P \neq |\tau^{(i+1)}|_P$ and thus the vectors $u^{(i)}$ and $u^{(i+1)}$ lie in different spaces $\mathcal{N}_{\tau^{(i)}}$ and $\mathcal{N}_{\tau^{(i+1)}}$, respectively.
We define the following map to convert between these spaces.

\begin{definition}\label{defn:projection}
For each pair of patterns $\pi,\tau \in P^*$, let $t = \dim(\mathcal{Z}_\tau)$ and $p = \dim(\mathcal{Z}_\pi)$, then we define the map $\mathrm{Proj}_{\pi,\tau}\colon \mathcal{Z}_\pi \to \mathcal{Z}_\tau$ such that
\[
	\mathrm{Proj}_{\pi,\tau}(u_1,u_2,\ldots,u_p)
	=
	(u_1,u_2,\ldots,u_p,0,0,\ldots,0)
\]
if $t > p$, and
\[
	\mathrm{Proj}_{\pi,\tau}(u_1,u_2,\ldots,u_p)
	=
	(u_1,u_2,\ldots,u_t)
\]
otherwise.
Notice that if $\dim(\mathcal{Z}_\tau) < \dim(\mathcal{Z}_\pi)$, then $\mathrm{Proj}_{\pi,\tau}$ is a projection; otherwise, $\dim(\mathcal{Z}_\tau) \geqslant \dim(\mathcal{Z}_\pi)$ and $\mathrm{Proj}_{\pi,\tau}$ is an embedding.
\end{definition}

In order to construct \cref{algo:word-shuffling}, we need to define a map which explicitly describes how to construct the sequence of extended patterned words in~(\ref{eq:patterned-sequence}).
We construct such a map in the following lemma.

\begin{lemma}\label{lemma:map-delta}
	We may construct a map
	\[
		\Delta \colon
		\textsc{StrPatt} \times W_1
		\to
		(\mathbb{N}_+ \cup \{\varnothing\}) \times \textsc{Patt} \times W_2,
	\]
	where
	\[
		W_1 = \{w \in S^* \mid 1 \leqslant |w|_S \leqslant d\}
		\quad\text{and}\quad
		W_2 = \{w \in S^* \mid |w|_S < d\}
	\]
	with the following properties.
	Let $((u,\tau),\sigma)$ be an extended strongly patterned word, and let $\Delta(\tau,w) = (x,\tau',w')$ with $w = \mathrm{Prefix}(\sigma)$.
	We may then \emph{apply} $\Delta$ to obtain an extended patterned word $((u',\tau'),\sigma')$ where $u' = \mathrm{Proj}_{\tau,\tau'}(u)+e_{\tau',x}$ and $\sigma' = (w\mapsto w') \cdot \sigma$.
	This will be denoted as
	\[
		((u,\tau),\sigma)
		\xrightarrow{\Delta}
		((u',\tau'),\sigma').
	\]
	For each extended strongly patterned word $((u,\tau),\sigma)$,
	\begin{enumerate}
		\item\label{lemma:map-delta:prop1}
			$|\tau|_P \leqslant |\tau'|_P$ and thus $\mathrm{Proj}_{\tau,\tau'}\colon\mathcal{N}_\tau \to \mathcal{N}_{\tau'}$ is an embedding;
		\item\label{lemma:map-delta:prop2}
			$u^\tau \sigma \simeq (u')^{\tau'}\sigma'$;
		\item\label{lemma:map-delta:prop3}
			$|\sigma|_S > |\sigma'|_S$ and $\omega(w) > \omega(w')$; and
		\item\label{lemma:map-delta:prop4}
			either $|\sigma'|_S = 0$, or $((u',\tau'),\sigma')$ is an extended strongly patterned word.
	\end{enumerate}
	Notice that property~\ref{lemma:map-delta:prop4} implies that either $((u',\tau'),\sigma')$ is equivalent to the patterned word $(u',\tau')$, or we may apply $\Delta$ again.
	From property~\ref{lemma:map-delta:prop3}, we see that after finitely many applications of the map $\Delta$, we have a patterned word.
\end{lemma}

\begin{proof}

Let $\tau = \tau_1 \tau_2 \cdots \tau_k \in P^*$ be a strong pattern, that is, $\tau$ is a pattern for which the coset representatives
\[
	\rho(\overline{\varepsilon}),\,
	\rho(\overline{\tau_1}),\,
	\rho(\overline{\tau_1 \tau_2}),\,
	\rho(\overline{\tau_1 \tau_2 \tau_3}),\,
	\ldots,\,
	\rho(\overline{\tau})
\]
are pairwise distinct.
Then, from the pigeonhole principle on the $d$ cosets of $\mathbb{Z}^n$ in $G$, we see that $|\tau|_P = k < d$.

Let $w \in S^*$ be a word with length $1 \leqslant |w|_S \leqslant d$.
We separate the remainder of this proof into the cases where $w \in P$ and $w \notin P$ as follows.

Suppose that $w \in P$, then we have a length $k+1$ pattern $\tau' = \tau w$,
moreover, from the definition of words in $P$, we see that $|w|_S < d$, and from \cref{rmk:short-prefix} we have $w = \sigma$.
We then define $\Delta(\tau,w) = (\varnothing, \tau', \varepsilon)$.
For each extended strongly patterned word $((u,\tau),w)$ with $u \in \mathbb{N}^p$, we then obtain an extended patterned word $((u',\tau'),\varepsilon)$ where $u' = \mathrm{Proj}_{\tau,\tau'}(u) = (u_1,u_2,\ldots,u_p,0,0,\ldots,0)$.
Notice that we have $(u')^{\tau'} = u^\tau w$.
This completes our proof for the case that $w \in P$.

In the remainder of this proof, we suppose that $w \notin P$.
From \cref{lemma:factoring-words}, we factor $w$ uniquely as $w = \alpha\beta\delta$ where $\alpha \in P \cup \{\varepsilon\}$, $\beta \in Y$ and $(|\alpha|_S,|\beta|_S)$ is minimal with respect to the lexicographic order on $\mathbb{N}^2$.
From the labelling $Y = \{y_1,y_2,\ldots,y_m\}$, we see that there must be an index $b$ such that $\beta = y_b$.

Let $((u,\tau),\sigma)$ be an extended strongly patterned word with $w = \mathrm{Prefix}(\sigma)$ and $u = (u_0, u_1, \ldots,u_k)$ where each $u_{a} = (u_{a,1}, u_{a,2},\ldots,u_{a,m}) \in \mathbb{N}^m$. 
Then if we factor $\sigma$ as $\sigma = w\zeta$, we see that
\begin{multline*}
	u^\tau \sigma
	=
	\Big(
		y_1^{u_{0,1}}
		y_2^{u_{0,2}}
		\cdots
		y_m^{u_{0,m}}
	\Big)
	\tau_1
	\Big(
		y_1^{u_{1,1}}
		y_2^{u_{1,2}}
		\cdots
		y_m^{u_{1,m}}
	\Big)
	\tau_2
	\\
	\cdots
	\tau_{k-1}
	\Big(
		y_1^{u_{k-1,1}}
		y_2^{u_{k-1,2}}
		\cdots
		y_m^{u_{k-1,m}}
	\Big)
	\tau_k
	\Big(
		y_1^{u_{k,1}}
		y_2^{u_{k,2}}
		\cdots
		y_m^{u_{k,m}}
	\Big)
	\alpha y_b \delta \zeta.
\end{multline*}
If there is an index $a$ with $0 \leqslant a \leqslant k$ such that $\rho(\overline{\tau_1 \tau_2 \cdots \tau_a}) = \rho(\overline{\pi \alpha})$, then the choice of such an index $a$ must be unique, and we see that
\[
	\overline{
		\tau_{a+1}
		\Big(
			y_1^{u_{a+1,1}}
			y_2^{u_{a+1,2}}
			\cdots
			y_m^{u_{a+1,m}}
		\Big)
		\tau_{a+2}
		\cdots
		\tau_k
		\Big(
			y_1^{u_{k,1}}
			y_2^{u_{k,2}}
			\cdots
			y_m^{u_{k,m}}
		\Big)
		\alpha
	} \in \mathbb{Z}^n
\]
commutes with $\overline{y_b} \in \mathbb{Z}^n$, that is,
\[
	{(u_0,\ldots,u_{a-1},u_a+e_b,u_{a+1},\ldots,u_k)}^\tau
	\alpha \delta \zeta
	\simeq
	u^\tau \alpha y_b \delta \zeta
	=
	u^\tau \sigma
\]
where $e_b \in \mathbb{N}^m$ is the $b$-th standard basis element.
In this case we define the map $\Delta(\tau,w) = (a\cdot m+b, \, \tau,\, \alpha\delta)$ and our proof is complete.
Otherwise, we see that the coset representatives
\[
	\rho(\overline{\varepsilon}),\,
	\rho(\overline{\tau_1}),\,
	\rho(\overline{\tau_1\tau_2}),\,
	\rho(\overline{\tau_1\tau_2\tau_3}),\,
	\ldots,\,
	\rho(\overline{\tau}),\,
	\rho(\overline{\tau\alpha})
\]
are pairwise distinct and $\alpha \neq \varepsilon$, that is, $\alpha \in P$.
Then we see that the length $k+1$ word $\tau' = \tau \alpha \in P^*$ is a strong pattern, and that we have
\[
	(u_0,u_2,\ldots,u_k,e_b)^{\tau'} \delta \zeta = u^\tau \alpha y_b \delta\zeta = u^\tau\sigma
\]
where $e_b \in \mathbb{N}^m$ is the $b$-th standard basis vector.
After defining $\Delta(\tau,w) = (a\cdot k+b, \tau', \delta)$ our proof is complete.
\end{proof}

We are now ready to define our algorithm as follows.

\begin{algorithm}[Word Shuffling]\label{algo:word-shuffling}
	Let $\Delta$ be the map in \cref{lemma:map-delta}.
	For each word $\sigma \in S^*$, there is a finite sequence of extended patterned words
	\begin{multline}\label{algo:word-shuffling/sequence}
		((\mathbf{0},\varepsilon),\sigma)
		=
		((u^{(1)},\tau^{(1)}),\sigma^{(1)})
		\xrightarrow{\Delta}
		((u^{(2)},\tau^{(2)}),\sigma^{(2)})
		\\\cdots
		\xrightarrow{\Delta}
		((u^{(q)},\tau^{(q)}),\sigma^{(q)})
		=
		((v,\pi),\varepsilon).
	\end{multline}
	From this sequence we define $\mathrm{Shuffle}(\sigma) = (v,\pi)$.
	Notice from property~\ref{lemma:map-delta:prop3} in \cref{lemma:map-delta} that we have
	\[
		|\sigma|_S = |\sigma^{(1)}|_S >
		|\sigma^{(2)}|_S >
		|\sigma^{(3)}|_S >
		\cdots >
		|\sigma^{(q)}|_S
		= 0
	\]
	and thus $q \leqslant |\sigma|_S + 1$.
From property~\ref{lemma:map-delta:prop2} in \cref{lemma:map-delta}, we see that $v^\pi \simeq \sigma$.
\end{algorithm}

In the remainder of this section, we compute the group elements and weights of patterned words, and determine which patterned word represents geodesics.

\subsection{Geodesic Patterned Words}\label{sec:patterned-words/properties}

From \cref{algo:word-shuffling}, for each word $\sigma \in S^*$ we have a well-defined patterned word $(v,\pi) = \mathrm{Shuffle}(\sigma)$ such that $v^\pi \simeq \sigma$, that is, $v^\pi$ represents the same group element as $\sigma$ with the same weight.
We see that $\sigma$ is a geodesic if and only if $v^\pi$ is a geodesic.

In this section, we modify an argument of Benson~\cite{benson1983} and show that the group element and weight of any word $v^\pi$ can be computed with the use of integer affine transforms, and that we may verify that $v^\pi$ is a geodesic by checking if the vector $v$ belongs to a polyhedral set $\mathcal{G}_\pi$.

\begin{lemma}\label{lemma:patterned-word-maps}
	For each pattern $\pi$, there are integer affine transformations $\Psi_{\pi} \colon \mathcal{Z}_\pi \to \mathbb{Z}^n$ and $\Omega_{\pi} \colon \mathcal{Z}_\pi \to \mathbb{Z}$ such that for each patterned word $(v,\pi)$, we have $\overline{v^\pi} = \Psi_\pi(v) \cdot \rho(\overline{\pi})$ and $\omega(v^\pi) = \Omega_\pi(v)$.
\end{lemma}

\begin{proof}

Recall that in \cref{defn:sets-Y-P} we fixed a labelling $Y = \{y_1, y_2, \ldots, y_m\}$ where $m  = |Y|$.
Define the matrix $Z \in \mathbb{Z}^{m \times n}$ such that $e_i Z = \overline{y_i}$ for each standard basis vector $e_i \in \mathbb{Z}^m$.
Then, we see that $vZ = \overline{y_1^{v_1}y_2^{v_2} \cdots y_m^{v_m}}$ for each $v \in \mathbb{N}^m$.
For each $p \in P$ we see that $\overline{p} x \overline{p}^{-1} \in \mathbb{Z}^n$ for each $x \in \mathbb{Z}^n \triangleleft G$; thus we define matrices $R_p \in \mathbb{Z}^{n \times n}$ such that $x R_p = \overline{p} x \overline{p}^{-1}$ for each $x \in \mathbb{Z}^n$.

To compute the element $\overline{v^\pi}$ we first rewrite $v^\pi$ as
\begin{multline*}
	\Big(
		y_1^{v_{1}}
		y_2^{v_{2}}
		\cdots
		y_m^{v_{m}}
	\Big)
	\cdot
	\pi_1
	\Big(
		y_1^{v_{m+1}}
		y_2^{v_{m+2}}
		\cdots
		y_m^{v_{2m}}
	\Big)
	\pi_1^{-1}
	\\
	(\pi_1 \pi_2)
	\Big(
		y_1^{v_{2m+1}}
		y_2^{v_{2m+2}}
		\cdots
		y_m^{v_{3m}}
	\Big)
	(\pi_1 \pi_2)^{-1}
	\\
	\cdots
	\pi
	\Big(
		y_1^{v_{km+1}}
		y_2^{v_{km+2}}
		\cdots
		y_m^{v_{(k+1)m}}
	\Big)
	\pi^{-1}
	\cdot
	\pi.
\end{multline*}
Then we see that $\rho(\overline{v^\pi}) = \rho(\overline{\pi})$ and $\psi(\overline{v^\pi}) = \Psi_{\pi}(v)$ where
\begin{multline*}
	\Psi_{\pi}(v)
	=
	(v_1,v_2,\ldots,v_m) Z +
	(v_{m+1},v_{m+2},\ldots,v_{2m}) Z R_{\pi_1} +\\
	\cdots +
	(v_{mk+1},v_{mk+2},\ldots,v_{(k+1)m}) Z R_{\pi_k} \cdots R_{\pi_2} R_{\pi_1}
	+ \psi(\pi).
\end{multline*}
Considering the word $v^\pi$ we see that $\omega(v^\pi) = \Omega_{\pi}(v)$ where
\[
	\Omega_{\pi}(v)
	=
	\omega(\pi) +
	\sum_{j=0}^k \sum_{i=1}^m
		v_{jm+i} \cdot \omega(y_i).
\]
The maps $\Psi_{\pi} \colon \mathcal{Z}_\pi \to \mathbb{Z}^n$ and $\Omega_{\pi} \colon \mathcal{Z}_\pi \to \mathbb{Z}$ are integer affine transforms.
\end{proof}

From the integer affine transformations defined in \cref{lemma:patterned-word-maps} and the closure properties of polyhedral sets we have the following result.

\begin{lemma}\label{lemma:geodesics-in-special-form}
	For each pattern $\pi$, there is a polyhedral set $\mathcal{G}_{\pi} \subseteq \mathcal{N}_\pi$ such that $v \in \mathcal{G}_{\pi}$ if and only if $(v,\pi)$ is a patterned word where $v^\pi$ is a geodesic.
\end{lemma}

\begin{proof}
From \cref{algo:word-shuffling} we see that the word $v^\pi$ is a geodesic if and only if there is no patterned word $(u,\tau)$ with $\overline{u^\tau} = \overline{v^\pi}$ and $\omega(u^\tau) < \omega(v^\pi)$.
For each pattern $\pi$, let $E_{\pi} \colon \mathcal{Z}_\pi \to \mathbb{Z}^{n+1}$ be the integer affine transformation defined as $E_{\pi} (v) = (\Psi_{\pi}(v),\Omega_{\pi}(v))$,
and let $\mathcal{R} \subseteq \mathbb{Z}^{2(n+1)}$ be the polyhedral set
\[
	\mathcal{R}
	=
	\left\{
		(\nu,\mu) \in \mathbb{Z}^{n+1} \times \mathbb{Z}^{n+1}
	\,\middle\vert\,
	\begin{aligned}
		\nu_1 = \mu_1,\,
		\nu_2 = \mu_2,\,
		\ldots,\,
		\nu_n = \mu_n\\
		\text{and }
		\nu_{n+1} > \mu_{n+1}
	\end{aligned}
	\right\}.
\]
Then, we see that $v^\pi$ is geodesic if and only if there is no patterned word $(u,\tau)$ with $\rho(\overline{\tau}) = \rho(\overline{\pi})$ and
$
	\big(E_{\pi}(v),E_{\tau}(u)\big)
	\in
	\mathcal{R}
$; or equivalently, $v^\pi$ is a geodesic if and only if the intersection
\[
	\Big(
		E_{\pi}(\{v\})
		\times
		E_{\tau}\!
			\left(\mathcal{N}_{\tau}\right)
	\Big)
	\cap
	\mathcal{R}
\]
is empty for each pattern $\tau$ with $\rho(\overline{\tau})=\rho(\overline{\pi})$.

Let $f \colon \mathbb{Z}^{n+1}\times \mathbb{Z}^{n+1} \to \mathbb{Z}^{n+1}$ be the projection onto the first $\mathbb{Z}^{n+1}$ factor, that is, $f(\nu,\mu)=\nu$ for each $(\nu,\mu) \in \mathbb{Z}^{n+1} \times \mathbb{Z}^{n+1}$.
Let
\[
	\mathcal{D}_{\pi,\tau}
	=
	\mathcal{N}_{\pi}
	\cap
	\left[
		\left(E_{\pi}\right)^{-1}
		f\left(
		\Big(
			E_{\pi}\!
			\left(\mathcal{N}_{\pi}\right)
			\times
			E_{\tau}\!
			\left(\mathcal{N}_{\tau}\right)
			\Big)
			\cap
			\mathcal{R}
		\right)
	\right]
.\]
Then, we see that $v^\pi$ is a geodesic if and only if $v \notin \mathcal{D}_{\pi,\tau}$ for each pattern $\tau$ with $\rho(\overline{\tau})=\rho(\overline{\pi})$.
Then, $v^\pi$ is a geodesic if and only if $v \in \mathcal{G}_{\pi}$ where
\[
	\mathcal{G}_{\pi}
	=
	\mathcal{N}_{\pi}
	\setminus
	\bigcup
	\Big\{
		\mathcal{D}_{\pi,\tau}
	\ \Big\vert\ 
		\tau
		\text{ is a pattern with }
		\rho(\overline{\tau})=\rho(\overline{\pi})
	\Big\}
\]
where we see that the above union is finite as there can be only finitely many patterns.
Moreover, from the closure properties in \cref{prop:affine-transforms-of-polyhedral-sets,prop:closure-properties-of-polyhedral-sets} we see that each set $\mathcal{G}_{\pi} \subseteq \mathcal{N}_\pi$ is polyhedral.
\end{proof}

\section{Geodesic Growth}\label{sec:geodesic-growth}


In this section we provide a characterisation of the geodesic growth of virtually abelian groups, in particular, we show that the geodesic growth of a virtually abelian group with respect to any finite weighted monoid generating set is either polynomial with rational geodesic growth series, or exponential with holonomic geodesic growth series.
This result is provided in \cref{thm:geodesic-growth}.

The result in \cref{thm:geodesic-growth} is proven by first showing that the geodesic growth series is holonomic, then applying the following lemma.

\begin{lemma}\label{lemma:holonomic-growth-series}
	If $G$ has a holonomic geodesic growth series with respect to the weighted monoid generating set $S$, then the geodesic growth is either exponential, or the polynomial of an integer degree with a rational geodesic growth series.
\end{lemma}

\begin{proof}
	The proof follows from \cref{cor:dichotomy-of-characterisation,lemma:holonomic-power-series-poles}.
\end{proof}

In \cref{lemma:path-to-special-form}, we construct a weight-preserving bijection from the words in $S^*$ to a subset of paths in a weighted graph $\Gamma$.
Then, in \cref{thm:geodesic-growth} we construct a weight-preserving bijection from the set of such paths which correspond to geodesics, and the set of words in a finite number of polyhedrally constrained languages.
Using the theory developed in \cref{sec:holonomic-functions}, we then prove our result.
We begin by constructing the graph $\Gamma$ as follows.

\begin{definition}\label{defn:graph-gamma}
	Let $\Delta$ be the map constructed in \cref{lemma:map-delta}.
	Let $\Gamma$ be the finite weighted directed edge-labelled graph defined as follows.
	For each pattern $\tau$ and each word $w \in S^*$ with $|w|_S \leqslant d = [G:\mathbb{Z}^n]$, the graph $\Gamma$ has a vertex $[\tau,w] \in \mathrm{V}(\Gamma)$.
	Suppose $\tau \in \textsc{StrPatt}$, $|w|_S \geqslant 1$ and that $\Delta(\tau,w) = (x,\tau',w')$;
	if $|w|_S = d$, then for each word $\xi \in S^*$ with $|w'\xi|_S \leqslant d$, the graph $\Gamma$ has a labelled edge $[\tau,w] \xrightarrow{x} [\tau',w'\xi]$;
	otherwise, $1 \leqslant |w|_S < d$ and the graph $\Gamma$ has a labelled edge $[\tau,w] \xrightarrow{x} [\tau',w']$.
	Moreover, each such edge has weight $\omega(w)-\omega(w') > 0$.
\end{definition}

We are interested in paths of the following form.

\begin{definition}\label{defn:path-sets}
	For each pattern $\pi$, we write $\textsc{Path}_\pi$ for the set of paths
	\[
	\textsc{Path}_\pi
	=
	\left\{
		p\colon
		[\varepsilon,w]
		\to^*
		[\pi, \varepsilon]
	\mid
		w \in S^* \text{ with }|w|_S \leqslant d
	\right\}.
	\]
	We write $\textsc{Path}$ for the union of all such sets, that is,
	$
		\textsc{Path}
		=
		\bigcup_\pi \textsc{Path}_\pi
	$.
\end{definition}

We count the instances of each edge label as follows.

\begin{definition}
	Let $\alpha \colon \textsc{Path} \to \bigcup\{ \mathcal{N}_\pi \mid \pi \in \textsc{Patt} \}$ map paths $p \in \textsc{Path}_\pi$ to vectors in $\mathcal{N}_\pi$ such that the $i$-th component of $\alpha(p)$ counts the number of edges of $p$ that are labelled with $i$, that is, if
	$
		p\colon
		\nu_1 \xrightarrow{x_1}
		\nu_2 \xrightarrow{x_2}
		\cdots \xrightarrow{x_k}
		[\pi,\varepsilon] \in \textsc{Path}_\pi,
	$
	then we have $\alpha(p) = v \in \mathcal{N}_\pi$ where each $v_i = \#\{j \mid x_j = i\}$.
\end{definition}

In the following lemma, we construct a weight-preserving bijection that maps from the set of words $S^*$ to the set of paths $\textsc{Path}$.

\begin{lemma}\label{lemma:path-to-special-form}
	We may construct a weight-preserving bijection $S^* \to \textsc{Path}$, which we denote as $\sigma \mapsto p_\sigma$, with the following properties.
	For each word $\sigma \in S^*$ where $\mathrm{Shuffle}(\sigma) = (v,\pi)$, we have $p_\sigma \in \textsc{Path}_\pi$.
	For each path $p \in \textsc{Path}_\pi$, there is a unique word $\sigma \in S^*$ such that $p = p_\sigma$ and $v^\pi \simeq \sigma$ where $v = \alpha(p) \in \mathcal{N}_\pi$.
\end{lemma}

\begin{proof}
Let $\sigma \in S^*$, then from \cref{algo:word-shuffling} there is a finite sequence
\begin{multline}\label{eq:delta-sequence}
	((\mathbf{0},\varepsilon),\sigma)
	=
	((u^{(1)},\tau^{(1)}),\sigma^{(1)})
	\xrightarrow{\Delta}
	((u^{(2)},\tau^{(2)}),\sigma^{(2)})
	\\\cdots
	\xrightarrow{\Delta}
	((u^{(q)},\tau^{(q)}),\sigma^{(q)})
	=
	((v,\pi),\varepsilon).
\end{multline}
Let  $w^{(j)} = \mathrm{Prefix}(\sigma^{(j)})$ and $\Delta(\tau^{(j)},w^{(j)}) = (x^{(j+1)},\tau^{(j+1)},w^{(j)\prime})$.

From \cref{lemma:map-delta} we see that each $\sigma^{(j+1)} = (w^{(j)}\mapsto w^{(j)\prime})\cdot \sigma^{(j)}$.
Then,
\begin{equation}\label{eq:path-to-word}
	\sigma
	=
	(w^{(1)\prime}\mapsto w^{(1)})
	(w^{(2)\prime}\mapsto w^{(2)})
	\cdots
	(w^{(q-1)\prime}\mapsto w^{(q-1)})
	\cdot
	\varepsilon
\end{equation}
and thus
\begin{multline}\label{eq:total-weight}
	\omega(\sigma)
	=
	(\omega(w^{(1)}) - \omega(w^{(1)\prime}))
	+ (\omega(w^{(2)}) - \omega(w^{(2)\prime}))\\
	+ \cdots
	+ (\omega(w^{(q-1)}) - \omega(w^{(q-1)\prime})).
\end{multline}
Moreover, from the properties of $\Delta$ given in \cref{lemma:map-delta}, and the definition of the graph $\Gamma$ given in \cref{defn:graph-gamma}, we see that
\begin{multline}\label{eq:path-p}
	p_\sigma \colon
	[\varepsilon,w]
	=
		[\tau^{(1)}, w^{(1)}]
	\xrightarrow{x^{(2)}}
		[\tau^{(2)}, w^{(2)}]
	\xrightarrow{x^{(3)}}\\
	\cdots
	\xrightarrow{x^{(q-1)}}
		[\tau^{(q-1)}, w^{(q-1)}]
	\xrightarrow{x^{(q)}}
		[\tau^{(q)}, w^{(q)}]
	= [\pi, \varepsilon]
\end{multline}
is a path in $\textsc{Path}_\pi$.
Notice that the weight of the path $p_\sigma$ is the same as the weight of $\sigma$ in (\ref{eq:total-weight}) and thus the map $\sigma \mapsto p_\sigma$ is weight preserving.
It remains to be shown that the map $\sigma \mapsto p_\sigma$ is a bijection.

Suppose that we are given a path $p_\sigma$ as in (\ref{eq:path-p}).
Then, we may recover the words $w^{(i)\prime}$ as $\Delta(\tau^{(j)},w^{(j)}) = (x^{(j+1)},\tau^{(j+1)},w^{(j)\prime})$.
Hence, we may recover the word $\sigma$ using equation (\ref{eq:path-to-word}).
Thus, we see that the map $\sigma \mapsto p_\sigma$ is one-to-one.
It remains to be shown that the map $\sigma \mapsto p_\sigma$ is onto, that is, for each $p \in \textsc{Path}$, there is a word $\sigma$ such that $p = p_\sigma$.

Let $p \in \textsc{Path}_\pi$ be a path written as 
\begin{multline*}
	p \colon
	[\varepsilon,w]
	=
	[\tau^{(1)}, w^{(1)}]
	\xrightarrow{x^{(2)}}
	[\tau^{(2)}, w^{(2)}]
	\xrightarrow{x^{(3)}}\\
	\cdots
	\xrightarrow{x^{(q-1)}}
	[\tau^{(q-1)}, w^{(q-1)}]
	\xrightarrow{x^{(q)}}
	[\tau^{(q)}, w^{(q)}]
	= [\pi, \varepsilon].
\end{multline*}
Let $\Delta(\tau^{(j)},w^{(j)}) = (x^{(j+1)},\tau^{(j+1)},w^{(j)\prime})$.
We define the words $\sigma^{(j)}$ such that
\[
	\sigma^{(j)}
	=
	(w^{(j)\prime}\mapsto w^{(j)})
	\cdot
	\sigma^{(j+1)}
\]
and $\sigma^{(q)} = \varepsilon$.
We define the vectors $u^{(j)} \in \mathcal{N}_{\tau^{(j)}}$ such that
\[
	u^{(j+1)}
	=
	\mathrm{Proj}_{\tau^{(j)},\tau^{(j+1)}}(u^{(j)})
	+ e_{\tau^{(j+1)},x^{(j+1)}}
\]
and $u^{(1)} = \mathbf{0} \in \mathcal{N}_{\tau^{(1)}}$.
From the property~\ref{lemma:map-delta:prop1} in \cref{lemma:map-delta} we see that each $|\tau^{(j)}|_P \leqslant |\tau^{(j+1)}|_P$, and thus from \cref{defn:projection} we see that
$u^{(q)} = \alpha(p)$.
Let $\sigma = \sigma^{(1)}$ and $v = u^{(q)}$, then we see that
\begin{multline*}
	((\mathbf{0},\varepsilon),\sigma)
	=
	((u^{(1)},\tau^{(1)}),\sigma^{(1)})
	\xrightarrow{\Delta}
	((u^{(2)},\tau^{(2)}),\sigma^{(2)})
	\\\cdots
	\xrightarrow{\Delta}
	((u^{(q)},\tau^{(q)}),\sigma^{(q)})
	=
	((v,\pi),\varepsilon).
\end{multline*}
From this, we see that $\mathrm{Shuffle}(\sigma) = (v,\pi)$ and that $p = p_\sigma$ with $\sigma \simeq v^\pi$ where $v = \alpha(p) \in \mathcal{N}_\pi$.
Moreover, we see that the map $\sigma \mapsto p_\sigma$ is onto.
\end{proof}

We may now prove our first main theorem as follows.

\setcounter{theoremx}{0}
\TheoremGeodesicGrowth

\begin{proof}

From \cref{lemma:path-to-special-form}, we may compute the geodesic growth function as
\begin{equation}\label{eq:geod-growth-sum}
	\gamma_S(k)
	=
	\sum_{\pi \in \textsc{Patt}}
	\#
	\{
		p \in \textsc{Path}_\pi
	\mid
		\omega(p) \leqslant k
		\text{ and }\alpha(p) \in \mathcal{G}_\pi
	\}
\end{equation}
where $\omega(p)$ is the weight of $p$, and $\mathcal{G}_\pi$ is the polyhedral set in \cref{lemma:geodesics-in-special-form}.
Notice that (\ref{eq:geod-growth-sum}) is a finite sum as we have only finitely many patterns.

Let $\Sigma$ be the weighted finite alphabet which contains the edges of $\Gamma$, that is, for each edge $e\colon \nu_1 \xrightarrow{x} \nu_2$ in $\Gamma$, there is a letter $(\nu_1,x,\nu_2) \in \Sigma$ with weight $\omega(e)$.
Then, for each pattern $\pi$, we have a weight-preserving bijection from the paths in $\textsc{Path}_\pi$ to the words in a language $L_\pi \subseteq \Sigma^*$, in particular, the language $L_\pi$ contains all words of the form
\[
	([\varepsilon,w],x_1,\nu_1)
	(\nu_1,x_2,\nu_2)
	(\nu_2,x_3,\nu_3)
	\cdots
	(\nu_k,x_{k+1},[\pi,\varepsilon]) \in \Sigma^*.
\]
Notice that each $L_\pi$ is a regular language.

We write $\Phi(\nu_1,x,\nu_2) \in \mathbb{N}^{|\Sigma|}$ to denote the Parikh vector corresponding to the letter $(\nu_1,x,\nu_2) \in \Sigma$.
For each pattern $\pi$, we define an integer affine transform $E_\pi \colon \mathbb{Z}^{|\Sigma|} \to \mathcal{Z}_\pi$ such that $E_\pi(\Phi(\nu_1,x,\nu_2)) = e_{\pi,x}$ is the $x$-th standard basis element for each $x \in \{1,2,\ldots,\dim(\mathcal{Z}_\pi)\}$, and $E_\pi(\Phi(\nu_1,x,\nu_2)) = \mathbf{0}$ otherwise.
Let $w \in L_\pi$ be the word corresponding to the path $p \in \textsc{Path}_\pi$, then we see that $\alpha(p) = E_\pi(\Phi(w))$.
From \cref{lemma:path-to-special-form}, we see that the path $p$ corresponds to a geodesic if and only if $\Phi(w) \in E^{-1}_\pi (\mathcal{G}_\pi)$.

For each pattern $\pi$, we define the constrained language $L^\mathrm{geod}_\pi \subseteq L_\pi$ as
\[
	L_\pi^\mathrm{geod}
	=
	\{
		w \in L_\pi
	\mid
		\Phi(w) \in E^{-1}(\mathcal{G}_\pi)
	\}.
\]
Notice that there is a weight-preserving bijection between $L^\mathrm{geod}_\pi$ and the set of geodesics $\sigma \in S^*$ with $p_\sigma \in \textsc{Path}_\pi$.
From \cref{prop:affine-transforms-of-polyhedral-sets} we see that $E^{-1}(\mathcal{G}_\pi)$ is a polyhedral set and thus each $L^\mathrm{geod}_\pi$ is a polyhedrally constrained language, as studied in \cref{sec:polyhedrally-constrained-languages}.
Then, from \cref{prop:polyhedrally-constrained-is-holonomic} we see that the multivariate generating function $f_\pi(x_1,x_2,\ldots,x_{|\Sigma|})$ of each $L^\mathrm{geod}_\pi$ is holonomic.

Let $a_{\pi,i}\in \mathbb{N}_+$ be the weight of the letter that corresponds to the variable $x_{i}$ in the generating function $f_\pi(x_1,x_2,\ldots,x_{|\Sigma|})$.
Let $h_\pi(z) \in \mathbb{C}[[z]]$ be defined as
\[
	h_\pi(z) =
	f_\pi(z^{a_{\pi,1}}, z^{a_{\pi,2}}, \ldots, z^{a_{\pi,|\Sigma|}}).
\]
Then we see that the coefficient of $z^k$ in $h_\pi(z)$ counts the geodesics $\sigma \in S^*$ for which $p_\sigma \in \textsc{Path}_\pi$ and $\omega(\sigma) = k$.
Let $g(z) \in \mathbb{C}[[z]]$ be defined as
\[
	g(z) = \frac{1}{1-z} \cdot \sum_{\pi \in \textsc{Patt}} h_\pi(z),
\]
Then we see that the coefficient of $z^k$ in $g(z)$ is given by $\gamma_S(k)$, that is, $g(z)$ is the geodesic growth series $g(z) = \sum_{k = 0}^\infty \gamma_S(k) z^k$.
Moreover, from the closure properties in \cref{lemma:holonomic-closure-properties} we see that the function $g(z)$ is holonomic.

Our result then follows from \cref{lemma:holonomic-growth-series}.
\end{proof}

\section{Language of Geodesics}\label{sec:language-of-geodesics}

In the previous section, we characterised the geodesic growth of virtually abelian groups.
In our proof of this result, we found a bijection between the geodesics of the virtually abelian group and a finite union of formal languages.
It is then natural to ask if there is a formal language characterisation for the language of geodesics.
In this section we show that the language of geodesics can be recognised by blind multicounter automaton.
Informally, we prove this result in \cref{thm:virtually-abelian-are-blind-counter} by implementing \cref{algo:word-shuffling} on a blind multicounter automaton, we then check if the word is geodesic using \cref{lemma:geodesics-in-special-form}.

\setcounter{theoremx}{2}
\TheoremBlindMulticounter

\begin{proof}

Let $G$ be a virtually abelian group that is generated as a monoid by a finite weighted set $S$, and let $\mathbb{Z}^n \triangleleft G$ with finite index $d = [G : \mathbb{Z}^n]$.
Let $\sigma \in S^*$, then from \cref{algo:word-shuffling} we have a patterned word $(v,\pi) = \mathrm{Shuffle}(\sigma)$ for which $v^\pi \simeq \sigma$ and thus $\sigma$ is a geodesic if and only if $v \in \mathcal{G}_{\pi}$ where $\mathcal{G}_\pi \subseteq \mathcal{N}_\pi$ is the polyhedral set given by \cref{lemma:geodesics-in-special-form}.

The idea of our proof is to simulate \cref{algo:word-shuffling} on a blind multicounter automaton, while maintaining enough information on the machine's counters so that we may verify the membership of the vector $v$ to the set $\mathcal{G}_{\pi}$.

For each polyhedral set $\mathcal{G}_\pi$, we fix a finite union of basic polyhedral sets
\[
	\mathcal{G}_{\pi}
	=
	\bigcup_{i=1}^{N_\pi}
	\mathcal{B}_{\pi,i}.
\]
Then, for each basic polyhedral set $\mathcal{B}_{\pi,i}$, we fix a finite intersection
\begin{multline}\label{eq:decompose-basic-poly}
	\mathcal{B}_{\pi,i}
	=
	\bigcap_{j=1}^{K_{\pi,i,1}}
		\left\{
			z \in \mathcal{Z}_\pi
		\,\middle\vert\,
			\alpha_{\pi,i,j} \cdot z > \beta_{\pi,i,j}
		\right\}
	\\\cap
	\bigcap_{j=1}^{K_{\pi,i,2}}
		\left\{
			z \in \mathcal{Z}_\pi
		\,\middle\vert\,
			\chi_{\pi,i,j} \cdot z \equiv \eta_{\pi,i,j}\ (\bmod\ \theta_{\pi,i,j})
		\right\}
	\\\cap
	\bigcap_{j=1}^{K_{\pi,i,3}}
		\left\{
			z \in \mathcal{Z}_\pi
		\,\middle\vert\,
			\xi_{\pi,i,j} \cdot z = \lambda_{\pi,i,j}
		\right\}
\end{multline}
where $\alpha_{\pi,i,j},\chi_{\pi,i,j},\xi_{\pi,i,j} \in \mathcal{Z}_\pi$, $\beta_{\pi,i,j},\eta_{\pi,i,j},\lambda_{\pi,i,j} \in \mathbb{Z}$ and $\theta_{\pi,i,j} \in \mathbb{N}_+$.

Let $k \in \mathbb{N}$ be such that $k \geqslant K_{\pi,i,1} + K_{\pi,i,2} + K_{\pi,i,3}$ for each basic polyhedral set $\mathcal{B}_{\pi,i}$.
In the remainder of this proof, we construct a blind $k$-counter automaton $M = (Q,S,\delta,q_0,F,\mathfrak{e})$ that recognises the language of geodesics.
Notice that the input alphabet of the machine is the generating set $S$.

For each basic polyhedral set $\mathcal{B}_{\pi,i}$, we define a map $C_{\pi,i} \colon \mathcal{N}_\pi \to \mathbb{Z}^k$ as
\begin{multline}\label{eq:configuration-of-counters}
	C_{\pi,i}(v) = (
		\alpha_{\pi,i,1} \cdot {v},\,
		\alpha_{\pi,i,2} \cdot {v},\,
		\ldots,\,
		\alpha_{\pi,i,K_{\pi,i,1}} \cdot {v},
	\\
		\chi_{\pi,i,1} \cdot {v},\,
		\chi_{\pi,i,2} \cdot {v},\,
		\ldots,\,
		\chi_{\pi,i,K_{\pi,i,2}} \cdot {v},\,
	\\
		\xi_{\pi,i,1} \cdot {v},\,
		\xi_{\pi,i,2} \cdot {v},\,
		\ldots,\,
		\xi_{\pi,i,K_{\pi,i,3}} \cdot {v},
		0,0,\ldots,0
	).
\end{multline}
Notice that a vector $v \in \mathcal{N}_{\pi}$ belongs to $\mathcal{B}_{\pi,i}$ if and only if
\[
	C_{\pi,i}(v)
	=
	(
		a_1, a_2,\ldots, a_{K_{\pi,i,1}},\,
		b_1, b_2,\ldots, b_{K_{\pi,i,2}},\,
		c_1, c_2,\ldots, c_{K_{\pi,i,3}},\,
		0,0,\ldots,0
	)
\]
where each $a_{j} > \beta_{\pi,i,j}$, each $b_j \equiv \eta_{\pi,i,j}\ (\bmod\ \theta_{\pi,i,j})$ and each $c_{j} = \lambda_{\pi,i,j}$.

\proofsection{State-Space of the Machine}

For each $\tau \in \textsc{Patt}$, each basic polyhedral set $\mathcal{B}_{\pi,i}$, and each word $w \in S^*$ with $|w|_S \leqslant d$, we have a state of the form $[\tau,w,\pi,i] \in Q$.
From these states, the machine will perform \cref{algo:word-shuffling} on its input word.

During the construction of our machine, we will ensure that if
\[
	(q_0,(0,0,\ldots,0),\sigma\mathfrak{e})
	\vdash^*
	([\tau,w,\pi,i],(c_1,c_2,\ldots,c_k),\zeta\mathfrak{e}),
\]
then there is a $u \in \mathcal{N}_\tau$ with $u^\tau w \zeta \simeq \sigma$ and $(c_1,c_2,\ldots,c_k) = C_{\pi,i}(\mathrm{Proj}_{\tau,\pi}(u))$.
In particular, this vector will correspond to some vector $u^{(i)}$ in the sequence of extended patterned words given in (\ref{algo:word-shuffling/sequence}) as constructed in \cref{algo:word-shuffling}.

For each basic polyhedral set $\mathcal{B}_{\pi,i}$, we have an accepting state $q_{\pi,i} \in F$.
Moreover, our construction will have the property that $(q_0,\mathbf{0},\sigma\mathfrak{e})\vdash^*(q_{\pi,i},\mathbf{0},\mathfrak{e})$ if and only if $\mathrm{Shuffle}(\sigma) = (v,\pi)$ with $v \in \mathcal{B}_{\pi,i}$,
and thus the machine $M$ will accept the word $\sigma$ if and only if it is a geodesic.

\proofsection{Nondeterministically Guessing a Basic Polyhedral Set}

The machine $M$ begins simulating the word shuffling algorithm after nondeterministically guessing a basic polyhedral set $\mathcal{B}_{\pi,i}$ for which $\mathrm{Shuffle}(\sigma) = (v,\pi)$ with $v \in \mathcal{B}_{\pi,i}$.
Notice that such a choice of basic polyhedral set exists if and only if $\sigma$ is a geodesic.
We accomplish this by introducing a relation
\[
	((q_0,\varepsilon),([\varepsilon,\varepsilon,\pi,i],\mathbf{0})) \in \delta
\]
for each basic polyhedral set $\mathcal{B}_{\pi,i}$,
that is, we have a transition
\begin{equation}\label{eq:first-transition}
	(q_0,(0,0,\ldots,0),\sigma\mathfrak{e})
	\vdash
	([\varepsilon,\varepsilon,\pi,i],(0,0,\ldots,0),\sigma\mathfrak{e})
\end{equation}
for each $\mathcal{B}_{\pi,i}$.
Notice that $\mathbf{0}^\varepsilon\sigma \simeq \sigma$ and $(0,0,\ldots,0) = C_{\pi,i}(\mathrm{Proj}_{\varepsilon,\pi}(\mathbf{0}))$.

\proofsection{Performing the Word Shuffling Algorithm}

For each extended strongly patterned word $((u^{(i)},\tau^{(i)}),\sigma^{(i)})$ in sequence (\ref{algo:word-shuffling/sequence}) in \cref{algo:word-shuffling}, we will see that $M$ has configurations of the form
\[
	([w,\tau^{(i)},\pi,i],C_{\pi,i}(u^{(i)}),\zeta)
\]
where $\sigma^{(i)} = w\zeta$.
In order to apply the map $\Delta$ from such a configuration we will require that $w = \mathrm{Prefix}(\sigma)$.
We do so by introducing transitions as follows.

Let $([\tau,w,\pi,i], (c_1, c_2, \ldots, c_k), \zeta\mathfrak{e})$ be a configuration of $M$ and let $\sigma = w \zeta$,
then $w = \mathrm{Prefix}(\sigma)$ if and only if either $|w|_S = d$ or $\zeta = \varepsilon$.
Thus, for each word $w \in S^*$ with $|w|_S < d$ and each $s \in S$, we introduce a relation of the form
\[
	(([\tau,w,\pi,i],s),([\tau,ws,\pi,i],\mathbf{0})) \in \delta
\]
for each $\tau,\pi,i$.
From these relations we have a unique partial computation
\begin{equation}\label{eq:finding-prefix}
	([\tau,w,\pi,i], (c_1, c_2, \ldots, c_k), \zeta\mathfrak{e})
	\vdash^*
	([\tau,w',\pi,i], (c_1, c_2, \ldots, c_k), \zeta'\mathfrak{e})
\end{equation}
where $w' = \mathrm{Prefix}(\sigma)$ and $\sigma = w\zeta = w'\zeta'$.
We then apply the map $\Delta$ as follows.

Let $\tau \in \textsc{StrPatt}$ be a strong pattern, let $w,\zeta \in S^*$ with $w = \mathrm{Prefix}(w\zeta)$ and $|w|_S \geqslant 1$, and let $\Delta(\tau,w) = (x,\tau',w')$.
From \cref{lemma:map-delta}, we see that for each vector $u \in \mathcal{N}_\tau$ we have $(u')^{\tau'}w'\zeta \simeq u^\tau w \zeta$ where $u' = \mathrm{Proj}_{\tau,\tau'}(u)+e_{\tau',x}$.
Moreover, we see that
\[
	C_{\pi,i}(\mathrm{Proj}_{\tau,\pi}(u')) =
		C_{\pi,i}(\mathrm{Proj}_{\tau,\pi}(u)) +
		C_{\pi,i}(\mathrm{Proj}_{\tau',\pi}(e_{\tau',x})).
\]
Notice that $w = \mathrm{Prefix}(w\zeta)$ if and only if either $|w|_S = d$ or $\zeta = \varepsilon$.
If $|w|_S = d$, then we introduce the relation
\[
	(([\tau,w,\pi,i],\varepsilon),([\tau',w',\pi,i],C_{\pi,i}(\mathrm{Proj}_{\tau',\pi}(e_{\tau',x})))) \in \delta
\]
for each $\pi,i$;
otherwise, if $|w|_S < d$, then we introduce the relation
\[
	(([\tau,w,\pi,i],\mathfrak{e}),([\tau',w',\pi,i],C_{\pi,i}(\mathrm{Proj}_{\tau',\pi}(e_{\tau',x})))) \in \delta
\]
for each $\pi,i$.
That is, we may apply the map $\Delta$ with the above relations.

Combining these transitions with those described in (\ref{eq:finding-prefix}), we see that after nondeterministically choosing a basic polyhedral set in (\ref{eq:first-transition}), the machine will deterministically perform \cref{algo:word-shuffling}, then enter a configuration of the form
\begin{equation}\label{eq:after-word-shuffle}
	(
		[\tau,\varepsilon,\pi,i],(c_1, c_2,\ldots,c_k), \mathfrak{e}
	)
\end{equation}
with $(c_1,c_2,\ldots, c_k) = C_{\pi,i}(v)$ where $(v,\tau) = \mathrm{Shuffle}(\sigma)$.

For each pair of patterns $\pi,\tau$ with $\pi \neq \tau$, and each basic polyhedral set $\mathcal{B}_{\pi,i}$, the machine has no transitions out of any configuration $([\tau,\varepsilon,\pi,i],c,\mathfrak{e})$ where $c \in \mathbb{Z}^k$.
Hence, if the computation enters such a state, it cannot continue to an accepting configuration.
Thus, we may assume without loss of generality that the machine nondeterministically chose the basic polyhedral set $\mathcal{B}_{\pi,i}$ with $\pi = \tau$ when performing the transition in (\ref{eq:first-transition}).
In the rest of our proof, we describe how the machine verifies that $v \in \mathcal{B}_{\pi,i}$.

\proofsection{Checking Polyhedral Set Membership}

Suppose that
\[
	(q_0,(0,0,\ldots,0),\sigma\mathfrak{e})
	\vdash^*
	([\pi,\varepsilon,\pi,i],(c_1,c_2,\ldots,c_k),\mathfrak{e}),
\]
then $(c_1,c_2,\ldots,c_k) = C_{\pi,i}(v)$ where $(v,\pi) = \mathrm{Shuffle}(\sigma)$.
For each state of the form $[\pi,\varepsilon,\pi,i]$, we introduce a relation
\[
	(([\pi,\varepsilon,\pi,i],\mathfrak{e}),(q_{\pi,i},\mu_{\pi,i})) \in \delta
\]
where
\begin{multline*}
	\mu_{\pi,i}
	=
	(
		-\beta_{\pi,i,1}-1,
		-\beta_{\pi,i,2}-1,
		\ldots,
		-\beta_{\pi,i,k}-1,
	\\
		-\eta_{\pi,i,1},
		-\eta_{\pi,i,2},
		\ldots
		-\eta_{\pi,i,k},
	\\
		-\lambda_{\pi,i,1},
		-\lambda_{\pi,i,2},
		\ldots
		-\lambda_{\pi,i,k},
		0,0,\ldots,0
	).
\end{multline*}
From this relation we have
\[
	([\pi,\varepsilon,\pi,i],(c_1,c_2,\ldots,c_k),\mathfrak{e})
	\vdash
	(q_{\pi,i},(c'_1,c'_2,\ldots,c'_k),\mathfrak{e})
\]
where $v \in \mathcal{B}_{\pi,i}$ if and only if $(c'_1,c'_2,\ldots,c'_k)$ belongs to the set
\[
	\mathbb{N}^{K_{\pi,i,1}} \times 
	\theta_{\pi,i,1}\mathbb{Z}
	\times \theta_{\pi,i,2}\mathbb{Z}
	\times \cdots
	\times \theta_{\pi,i,K_{\pi,i,2}}\mathbb{Z}
	\times \{0\}^{k-K_{\pi,i,1}-K_{\pi,i,2}}.
\]
We verify $v$'s membership to $\mathcal{B}_{\pi,i}$ by introducing additional relations as follows.
For each $1 \leqslant j \leqslant K_{\pi,i,1}$, we have
\[
	((q_{\pi,i},\mathfrak{e}),(q_{\pi,i},-e_j)) \in \delta,
\]
where $e_j \in \mathbb{Z}^k$ is the $j$-th standard basis element, and for each $ 1 \leqslant j \leqslant  K_{\pi,i,2}$
\[
	((q_{\pi,i},\mathfrak{e}),(q_{\pi,i},\pm\theta_{\pi,i,j} \, e_{j'})) \in \delta
\]
where $j' = K_{\pi,i,1}+j$ and $e_{j'} \in \mathbb{Z}^k$ is the $j'$-th standard basis element.
From these relations, we see that we see that
\[
	(q_0,(0,0,\ldots,0),\sigma\mathfrak{e})
	\vdash^*
	(q_{\pi,i},(0,0,\ldots,0),\mathfrak{e}),
\]
if and only if $v \in \mathcal{B}_{\pi,i}$ where $(v,\pi) = \mathrm{Shuffle}(\sigma)$.
\end{proof}

\section{Concluding Remarks}\label{sec:virt-abel/concluding}

In this chapter we characterised the geodesic growth for all virtually abelian groups with respect to every generating set.
Moreover, the proofs in this chapter are constructive, i.e., it is possible to compute the geodesic growth series for any given virtually abelian group.

\Citeauthor{bridson2012} gave a sufficient condition for a virtually abelian groups to have polynomial geodesic growth (see \cref{lem:bbes-main-theorem}).
It would be interesting to see if this condition is also necessary, that is, we ask the following question.

\begin{question}
	Let $G$ be a virtually abelian group with polynomial geodesic growth.
	Then, is there an element $g \in G$ whose normal closure is a finite-index abelian subgroup of $G$?
\end{question}

Recall that the geodesic growth is bounded from below by the volume growth, and thus, any group with polynomial geodesic growth must be virtually nilpotent.
In this chapter, we have shown that for any abelian group $A$, there is a virtually-$A$ group with polynomial geodesic growth (see the presentation in \cref{eq:virt-abel-polynomial}); and we have provided a method to determine if a virtually abelian group has polynomial geodesic growth.
Until now, the only known examples of polynomial geodesic growth have been virtually abelian, e.g., the groups studied by \textcite{bridson2012}.
It is then natural to ask if every group with polynomial geodesic growth is virtually abelian, and if not, then what counterexamples are there.
In \cref{chapter:virtually-heisenberg}, we take the next step towards answering this question, and in moving forward to a classification of polynomial geodesic growth.