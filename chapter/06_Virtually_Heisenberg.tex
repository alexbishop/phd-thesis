\chapter{Towards Virtually Nilpotent Groups}\label{chapter:virtually-heisenberg}

In \cref{chapter:polynomial-geodesic-growth} we characterised the geodesic growth of virtually abelian groups.
In this chapter, we take the next step towards a classification of polynomial geodesic growth by furnishing an example of a virtually 2-step nilpotent group with polynomial geodesic growth.
This is the first group which has been shown to have polynomial geodesic growth and is not virtually abelian.
This result is important as it shows that a classification of polynomial geodesic growth must include groups beyond the class of virtually abelian.

In \cref{thm:main}, we show that there is a group that is virtually 2-step nilpotent and has polynomial geodesic growth.
Our proof relies on a result that is implicit in the work of Blach\`ere~\cite{blachere2003} which we provide in Lemma~\ref{lem:heisenberg-geodesic}.

This result shows that there are groups with subexponential geodesic growth which are not virtually abelian, in particular, this example opens the door to the possible existence of a virtually nilpotent group that has intermediate geodesic growth with respect to some generating set.
It also raises the question of whether polynomial geodesic growth is restricted to virtually nilpotent groups of step at most 2.

\section{A Virtually Heisenberg Group}

The integer Heisenberg group is the group of $3 \times 3$ upper-triangular integer matrices with 1's on their diagonals, that is, the group generated by the matrices
\[
	a
	=
	\begin{bmatrix}
		1 & 1 & 0\\
		0 & 1 & 0\\
		0 & 0 & 1
	\end{bmatrix}
	\quad\text{and}\quad
	b
	=
	\begin{bmatrix}
		1 & 0 & 0\\
		0 & 1 & 1\\
		0 & 0 & 1
	\end{bmatrix}.
\]
It is well known that the integer Heisenberg group, $\Heisenberg$, has the presentation
\[
	\Heisenberg =
	\left\langle
		a,b
	\mid
		[a,[a,b]] = [b,[a,b]] = 1
	\right\rangle.
\]
Let $X$ denote the standard generating set $X = \{a,a^{-1},b,b^{-1}\}$ for $\Heisenberg$, following the convention of \textcite{blachere2003} we write $(x,y,z) \in \Heisenberg$ for the element corresponding to the normal form word $[a,b]^z b^y a^x$.
Inspired by the polynomial geodesic growth example in \cref{eq:virtually-z2-example}, we construct a virtually Heisenberg group $\vH$ as follows.
\begin{equation}\label{eq:presentation-1}
	\vH =
	\left\langle
		a,b,t
	\mid
		[a,[a,b]] = [b,[a,b]] = t^2 = 1,\ a^t = b
	\right\rangle.
\end{equation}
From the relation $b = a^t$ and applying a Tietze transform, we see that $S = \{a,a^{-1},t\}$ is a generating set for $\vH$.
We provide a partial picture of the Cayley graph of $\vH$ in Figure~\ref{fig:HeisCG}. Informally, one may think of this group as two copies of $\Heisenberg$ ``glued'' together with a ``twist'' by $t$ edges.

\begin{figure}[!ht]
	\centering
	\includegraphics{figure/virtuallyHeisenberg}
	\caption{A Cayley graph for $\vH$ with respect to the generating set $S$ where the undirected edges are labelled by $t$ and directed edges labelled by $a$.}\label{fig:HeisCG}
\end{figure}

Our goal is to show that any geodesic of $\vH$ with respect to the generating set $S$ can contain at most $7$ instances of the letter $t$.
From this we are able to place a polynomial upper bound on the geodesic growth function of $\vH$.
To do this, we first study geodesics of the integer Heisenberg group with respect to the generating set $X$.

\Textcite[Theorem~2.2]{blachere2003} provided explicit formulae for the length of elements in $\Heisenberg$, with respect the generating set $X$, by constructing geodesic representatives.
We provide the following lemma which is implicit in the proof of Theorem~2.2 in \cite{blachere2003}.

\begin{lemma}\label{lem:heisenberg-geodesic}
	Each element $(x,y,z) \in \Heisenberg$ has a geodesic representative with respect to the generating set $X = \{a,a^{-1},b,b^{-1}\}$ of the form
	\[
		a^{\alpha_1}
		b^{\beta_1}
		a^{\alpha_2}
		b^{\beta_2}
		a^{\alpha_3}
		b^{\beta_3}
	\quad
	\mathrm{or}
	\quad
		b^{\beta_1}
		a^{\alpha_1}
		b^{\beta_2}
		a^{\alpha_2}
		b^{\beta_3}
		a^{\alpha_3}
	\]
	where each $\alpha_i,\beta_j \in \mathbb{Z}$.
\end{lemma}

\begin{proof}
	We see that the lemma holds in the case of $(0,0,0) \in \Heisenberg$ as the empty word $\varepsilon \in S^*$ is such a geodesic.
	In the remainder of the proof, we assume that $(x,y,z) \neq (0,0,0)$.
	Following Blach\`ere~\cite[p.~22]{blachere2003} we reduce this proof to the case where $x,z \geq 0$ and $-x \leq y \leq x$ as follows.
	
	Let $\tau \colon X^* \to X^*$ be the monoid isomorphism defined such that $\tau(a^k)=b^k$ and $\tau(b^k)=a^k$ for each $k \in \mathbb{Z}$.
	Let $w^R$ denote the \emph{reverse} of the word $w$, that is, if $w = w_1 w_2 \cdots w_k$  where each $w_j \in X$, then $w^R = w_k \cdots w_2 w_1$.
	
	If $w \in X^*$ is a word as described in the lemma statement with $\overline{w} = (x,y,z)$, then $w'=  \tau(w^R)$ is also in the form described in the lemma statement and $\overline{w'} = (y,x,z)$.
	Moreover, we see that $\tau(w^R)$
	is a geodesic if and only if $w$ is a geodesic.
	Defining the monoid isomorphisms $\varphi_a,\varphi_b\colon X^* \to X^*$ by $\varphi_a(a^k) = a^{-k}$, $\varphi_a(b^k) = b^k$, and $\varphi_b(a^k) = a^{k}$, $\varphi_b(b^k) = b^{-k}$ for each $k \in \mathbb{Z}$, we see that if $w \in X^*$ is a geodesic representative for $(x,y,z) \in \Heisenberg$, then $\varphi_a(w)$, $\varphi_b(w)$ and $\varphi_a(\varphi_b(w))$ are geodesics for $(-x,y,-z)$, $(x,-y,-z)$ and $(-x,-y,z)$, respectively, and each such word is in the form as described in the lemma statement.
	From application of the above transformations, we may assume without loss of generality that $x,z \geq 0$ and $-x \leq y \leq x$.
	
	Let $h = (x,y,z) \in \Heisenberg$, then from \cite[Theorem~2.2]{blachere2003} we have the following formulae for the length $\ell_X(h)$ and (most importantly for us) geodesic representative for $h$.
	
	\begin{itemize}
		\item[I.]
		If $y \geq 0$, then we have the following cases.
		\begin{itemize}
			\item[I.1.]
			If $x < \sqrt{z}$, then $\ell_X(h) = 2\lfloor\,2\sqrt{z}\, \rfloor - x - y$
			and $h$ has a geodesic representative given by $b^{y-y'} S_z a^{x-x'}$ where $x',y'$ are the values given by $\overline{S_z} = (x',y',z)$ (cf.~\cite[p.~32]{blachere2003}), where $S_z$ is as follows.
			
			\begin{itemize}
				\item
				If $z = (n+1)^2$ for some $n \in \mathbb{N}$, then $S_z = a^{n+1} b^{n+1}$;
				\item 
				if there exists a $k \in \mathbb{N}$ with $1 \leq k \leq n$ such that $z = n^2 + k$, then let $S_z = a^k b a^{n-k} b^n$;
				\item
				otherwise, there exists some $k \in \mathbb{N}$ with $1 \leq k \leq n$ such that $z = n^2+n+k$ and we have $S_z = a^k b a^{n+1-k} b^n$.
			\end{itemize}
			
			
			\item[I.2.]
			If $x \geq \sqrt{z}$, then we have the following two cases:
			\begin{itemize}
				\item[I.2.1] $xy \geq z$, then $\ell_X(h) = x+y$, otherwise
				\item[I.2.2] $xy \leq z$, then $\ell_X(h) = 2 \lceil z/x \rceil + x - y$;
			\end{itemize}
			and in both cases, the word $b^{y-u-1} a^v b a^{x-v} b^u$ is a geodesic for $h$ where $0 \leq u$, $0 \leq v < x$ and $z = ux+v$ (cf.~pages~24,\,32 and 33 in \cite{blachere2003}).
		\end{itemize}
		
		\item[II.]
		If $y < 0$, then we have the following cases.
		\begin{itemize}
			\item[II.1.] If $x \leq \sqrt{z - xy}$, then $\ell_X(h) = 2\lceil\, 2\sqrt{z-xy}\,\rceil - x + y$.
			Let $n = \lceil \, \sqrt{z-xy}\,\rceil-1$.
			Then
			\begin{itemize}
				\item 
				there is either some $k \in \mathbb{N}$ with $1 \leq k \leq n$ such that we have $z-xy = n^2+k$, and $h$ has  $a^{x-n} b^{-n-1} a^k b a^{n-k} b^{n+y}$ as a geodesic representative; or
				\item
				there is some $k \in \mathbb{N}$ with $0 \leq k \leq n$ such that we have $z-xy = (n+1)^2-k$ and $a^{x-n} b^{-k} a^{-1} b^{k-n-1} a^{n+1} b^{n+1+y}$ is a geodesic representative for $h$
				(cf.~\cite[p.~24]{blachere2003}\footnote{Note that in \cite{blachere2003} there is an error in the second case.}).
			\end{itemize}
			%
			\item[II.2.] If $x \geq \sqrt{z - xy}$, then $\ell_X(h) = 2 \lceil z/x \rceil + x - y$ and 
			$h$ has  a geodesic representative of $b^{y-u-1} a^v b a^{x-v} b^u$
			where $u,v \geq 0$, $v < x$ and $z = ux+v$ (cf.~\cite[pp.~24\,\&\,33]{blachere2003}).
		\end{itemize}
	\end{itemize}
	Notice that in each of the above cases, we have our desired result.
\end{proof}

From this lemma, we  have the following result. 

\begin{corollary}\label{cor:max-7-t}
	If $w \in S^*$ is a geodesic of $\vH$ with respect to the generating set $S = \{a,a^{-1},t\}$, then $w$  contains at most 7 instances of the letter $t$.
\end{corollary}

\begin{proof}
	Let $w \in S^*$ be a word containing $8$ instances of $t$ of the form
	\[
	w
	=
	a^{n_1} t 
	a^{m_1} t
	a^{n_2} t 
	a^{m_2} t
	a^{n_3} t 
	a^{m_3} t
	a^{n_4} t 
	a^{m_4} t,
	\] where $n_i,m_i\in \mathbb Z$, 
	and notice that $\overline{w}$ belongs to the subgroup $\Heisenberg$.
	The Tietze transform given by $b = tat$ which we applied to obtain the generating set $S = \{a,a^{-1},t\}$ from \eqref{eq:presentation-1} yields an automorphism $\varphi \colon \vH \to \vH$ given by $\varphi(a) = a$, $\varphi(t) = t$, $\varphi(b) = tat$, and since 
	$t^2 = 1$ 
	we have $\varphi(b^k) = ta^kt$ for  $k \in \mathbb{Z}$.
	Let $X = \{a,a^{-1},b,b^{-1}\}$ be a generating set for the subgroup $\Heisenberg$.
	Then from the word $w \in S^*$ we may construct a word 
	\[
	w_2
	=
	a^{n_1}
	b^{m_1}
	a^{n_2}
	b^{m_2}
	a^{n_3}
	b^{m_3}
	a^{n_4}
	b^{m_4}\in X^*
	\]
	where $\overline{w_2} = \overline{w}$ since $\varphi(w_2) = w$.
	Moreover, $|w|_S = |w_2|_X + 8$.
	
	From Lemma~\ref{lem:heisenberg-geodesic}, we know that there is a word $w_3 \in X^*$, with  $\overline{w_3}=\overline{w_2}$ and $|w_3|_X \leq |w_2|_X$, of the form
	\[
	w_3
	=
	a^{\alpha_1}
	b^{\beta_1}
	a^{\alpha_2}
	b^{\beta_2}
	a^{\alpha_3}
	b^{\beta_3}
	\ \ \text{or}\ \ 
	w_3
	=
	b^{\beta_1}
	a^{\alpha_1}
	b^{\beta_2}
	a^{\alpha_2}
	b^{\beta_3}
	a^{\alpha_3}
	\] where $\alpha_i,\beta_i\in\mathbb Z$ (possibly zero).
	We then see that $\overline{w}$ can be represented by a word of the form
	\[
	w_4
	=
	a^{\alpha_1} t
	a^{\beta_1} t
	a^{\alpha_2} t
	a^{\beta_2} t
	a^{\alpha_3} t
	a^{\beta_3} t
	\ \ \text{or}\ \ 
	w_4
	=
	t
	a^{\beta_1} t
	a^{\alpha_1} t
	a^{\beta_2} t
	a^{\alpha_2} t
	a^{\beta_3} t
	a^{\alpha_3}
	\]
	where \[|w_4|_S = |w_3|_X + 6 < |w_2|_X + 8 = |w|_S.\]
	
	Then $w$ cannot be a geodesic as we have a strictly shorter word $w_4$ that represents the same element.
	Thus, a geodesic of $\vH$ with respect to $S = \{a,a^{-1},t\}$ can contain at most $7$ instances of the letter $t$ as we can replace any subword with $8$ instances of $t$ with a strictly shorter word containing at most $7$ instances of $t$.
\end{proof}

From this corollary we may immediately obtain the following polynomial upper bound on the geodesic growth function.

\setcounter{theoremx}{1}
\TheoremVirtuallyHeisenberg

\begin{proof}
	From Corollary~\ref{cor:max-7-t}, we see that any geodesic of $\vH$, with respect to the generating set $S$, must have the form
	\[
	w = a^{m_1} t a^{m_2} t \cdots t a^{m_{k+1}}
	\]
	where $k \leq 7$ and each $m_i \in \mathbb{Z}$.
	Then with $k$ fixed and $r = |w|_S$, we see that there are at most $2^{k+1}$ choices for the sign of $m_1,m_2,\ldots,m_{k+1}$, and at most $\binom{r}{k}$ choices for the placement of the $t$'s in $w$.
	Thus, the geodesic growth function $\gamma_S(n)$ has an upper bound given by
	\[
		\gamma_S(n)
		\leq
			\sum_{k=0}^7
			\sum_{r=k}^n
			2^{k+1}
			\binom{r}{k}
		\leq
			\sum_{k=0}^7
			\sum_{r=k}^n
			2^{k+1}
			r^k
		\leq
			8 \cdot 2^8 n^{8}
	\]
	which gives the degree 8 polynomial upper bound.
\end{proof}

\section{Concluding Remarks and Open Questions}\label{sec:virt-heisenberg/concluding}

The proof that the virtually 2-step nilpotent group in this chapter has polynomial geodesic growth relied heavily on work of \citeauthor{blachere2003} (see~\cref{lem:heisenberg-geodesic}) and does not appear to be generalisable in its current form.
It is then natural to ask if there are nilpotent groups of higher step with analogous properties, in particular, we ask the following question.

\begin{question}
	Is there a virtually $k$-step nilpotent group with polynomial geodesic growth for some $k \geqslant 3$, and if so, is there such an example for each $k$?
\end{question}

It follows from the work of \textcite[Theorem~2]{bass1972} that the usual growth rate of a virtually nilpotent group is polynomial of integer degree.
Moreover, from \cref{thm:geodesic-growth} it is known that if a virtually abelian group has polynomial geodesic growth, then it must be of integer degree since the geodesic growth series is rational in this case.
It is not known if there is a group with polynomial geodesic growth of a non-integer degree.

\begin{question}
	Is there a group with polynomial geodesic growth of a non-integer degree?
\end{question}

Based on experimental results (see~\cite{githubcode}) we conjecture that the geodesic growth rate of $\vH$ with respect to the generating set $S$ can be bounded from above and below by polynomials of degree six (cf.~the volume growth is polynomial of degree four).
We ask the following question.

From \cref{thm:geodesic-growth} we know that if a virtually abelian group has polynomial geodesic growth, then its geodesic growth series is rational.
However, it is unclear if this property is held by virtually nilpotent groups.
From experimental results, it appears that the geodesic growth series of $\vH$ with respect to $S$ is not rational (see~\cite{githubcode}).

\begin{question}
	Is the geodesic growth series for $\vH$ with respect to $S$ rational?
\end{question}

In this thesis we have taken steps towards a classification of polynomial geodesic growth, and more generally towards the study of the geodesic growth of virtually nilpotent groups.
In particular, we characterised the geodesic growth of virtually abelian groups; and provided the first example of a group with polynomial geodesic growth that is not virtually abelian.
The results in this thesis open up new questions and new techniques for obtaining characterisations.