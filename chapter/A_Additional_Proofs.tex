\chapter{Additional Proofs}

In this appendix we provide proofs of incidental statements made throughout this thesis.
These proofs are provided in the interest of completeness.

In the introduction, it was stated that the word problem for any finitely presented group is recursively enumerable.
This is a well-known fact which we prove as follows.

\begin{proposition}\label{prop:appendix/wp-fp}
	Finitely presented groups have recursively enumerable word problems.
\end{proposition}

\begin{proof}
	Let $G$ be a group with presentation $\left\langle X \mid R \right\rangle$ where $X$ and $R$ are both finite.
	We then see that a word $w \in X^*$ is in the word problem if and only if we have
	\[
		w
		=_{F_X}
		\prod_{i=0}^n  u_i r_i^{\delta_i} u_i^{-1}
	\]
	for some $n \in \mathbb{N}$ where each $u_i \in X^*$, $\delta_j \in \{-1,1\}$ and $r_i \in R$.
	Notice here that `$=_{F_X}$' denotes that the left and right-hand sides are the same word after free-reduction is performed, that is, they are equivalent if viewed as elements of the free group.
	
	We may then construct a Turing machine $M$ which takes a word $w \in X^*$ as input, then iterates through the set of all finite products $\prod_{i=0}^n  u_i r_i^{\delta_i} u_i^{-1}$.
	At each iteration, the machine should compare the word $w$ and the result of the product.
	The machine then terminate and accepts only if the two words are equal.
	
	We see that the machine $M$ accepts a word $w \in X^*$ if and only if it lies within the word problem $\WP_X$, that is, membership to the word problem $\WP_X$ is \emph{semi-decidable}.
	
	It is well known that a problem is semi-decidable if and only if the set of all accepted words (in this case the word problem) is recursively enumerable.
	This can be proven by constructing a machine which checks all words in parallel using a technique known as \emph{dovetailing} (see Theorem 20.8 on p.~441 of \cite{rich2007} for a proof of this fact).
\end{proof}

In \cref{sec:linear-constraints} we gave an example of a linearly constrained language, and provided its generating function.
In the following we show that this generating function is holonomic by explicitly constructing a system of linear differential equations which it satisfies.

\begin{proposition}\label{prop:appendix/holonomic-function}
	The multivariate power series
	\[
		f(x,y,z) =
		\sum_{n \in \mathbb{N}}
		\frac{(3n)!}{(n!)^3}
		x^n y^n z^n
	\]
	satisfies the differential equations
	\begin{equation}\label{eq:appendix/holonomic-differential-eq}
	\left.
	\begin{aligned}
		(x^2 - 27 x^3 y z) \partial_x^2 f(x,y,z)
		+ (x - 54 x^2 y z) \partial_x f(x,y,z)
		- 6 x y z \, f(x,y,z)
		&= 0
		\\
		(y^2 - 27 x y^3 z) \partial_y^2 f(x,y,z)
		+ (y - 54 x y^2 z) \partial_y f(x,y,z)
		- 6 x y z \, f(x,y,z)
		&= 0
		\\
		(z^2 - 27 x y z^3) \partial_z^2 f(x,y,z)
		+ (z - 54 x y z^2) \partial_z f(x,y,z)
		- 6 x y z \, f(x,y,z)
		&= 0.
	\end{aligned}
	\right\}
	\end{equation}
	Thus, $f(x,y,z)$ is holonomic.
\end{proposition}

\begin{proof}
	Notice that the system of differential equations in \eqref{eq:appendix/holonomic-differential-eq} is equivalent to
	\begin{equation}\label{eq:appendix/holonomic-differential-eq2}
	\left.
	\begin{aligned}
		x^2 \partial_x^2 f(x,y,z)
		+ x \partial_x f(x,y,z)
		-27 x^2 \partial_x^2 (x y z \, f(x,y,z))
		- 6 x y z \, f(x,y,z)
		&= 0
		\\
		y^2 \partial_y^2 f(x,y,z)
		+ y \partial_y f(x,y,z)
		-27 y^2 \partial_y^2 (x y z \, f(x,y,z))
		- 6 x y z \, f(x,y,z)
		&= 0
		\\
		z^2 \partial_z^2 f(x,y,z)
		+ z \partial_z f(x,y,z)
		-27 z^2 \partial_z^2 (x y z \, f(x,y,z))
		- 6 x y z \, f(x,y,z)
		&= 0.
	\end{aligned}
	\right\}
	\end{equation}
	This can be shown using the \emph{product rule} of differentiation.
	
	Let
	\[
		f(x,y,z) =
		\sum_{n \in \mathbb{N}}
		\frac{(3n)!}{(n!)^3}
		x^n y^n z^n,
	\]
	then
	\begin{align*}
			x \partial_x f(x,y,z)
			&=
			\sum_{n=1}^\infty
			\frac{(3n)!}{(n!)^3}
			\,n\,
			x^n y^n z^n,
		\\
			x^2 \partial_x^2 f(x,y,z)
			&=
			\sum_{n=1}^\infty
			\frac{(3n)!}{(n!)^3}
			\,n(n-1)\,
			x^n y^n z^n,
		\\
			xyz \, f(x,y,z)
			&=
			\sum_{n=1}^\infty
			\frac{(3(n-1))!}{((n-1)!)^3}
			\,
			x^n y^n z^n,\text{ and}
		\\
			x^2 \partial_x^2 (xyz \, f(x,y,z))
			&=
			\sum_{n=1}^\infty
			\frac{(3(n-1))!}{((n-1)!)^3}
			\,n(n-1)\,
			x^n y^n z^n.
	\end{align*}
	
	We then see that
	\[
		x^2 \partial_x^2 f(x,y,z)
		+ x \partial_x f(x,y,z)
		-27 x^2 \partial_x^2 (x y z \, f(x,y,z))
		- 6 x y z \, f(x,y,z)
		=
		\sum_{n=1}^\infty c_n x^n y^n z^n
	\]
	where each
	\begin{align*}
		c_n
		&=
		n^2
		\frac{(3n)!}{(n!)^3}
		-
		(27n^2 - 27n + 6)
		\frac{(3(n-1))!}{((n-1)!)^3}
		\\
		&=
		\frac{1}{n}
		\left[
			n^3
			\frac{(3n)!}{(n!)^3}
			-
			3n(3n-1)(3n-2)
			\frac{(3(n-1))!}{((n-1)!)^3}
		\right].
	\end{align*}
	Moreover, we see that each $c_n = 0$ as
	\[
		\frac{(3n)!}{(n!)^3}
		=
		\frac{3n(3n-1)(3n-2)}{n^3}
		\cdot
		\frac{(3(n-1))!}{((n-1)!)^3}.
	\]
	Thus, we have
	\[
		x^2 \partial_x^2 f(x,y,z)
		+ x \partial_x f(x,y,z)
		-27 x^2 \partial_x^2 (x y z \, f(x,y,z))
		- 6 x y z \, f(x,y,z)
		=
		0.
	\]
	The proofs of the other two differential equations in \eqref{eq:appendix/holonomic-differential-eq2} are the same.
\end{proof}
